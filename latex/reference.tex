\section{Reference}
\label{reference}

\subsection{\texttt{bmotion.json} Manifest}
\label{sec:manifest}

The options (and defaults) are:

\subsubsection{name (Type: String, Default value: MyVisualization)}

The name of the visualization.

\subsubsection{template (Type: String, Default value: template.html)}

The relative path to the HTML visualization file  (e.g. ``template.html'').

\subsubsection{model (Type: String, \textit{Required})}

The relative path to the model file (e.g. ``model/mymodel.mch'').

\subsubsection{tool (Type: String, Default value: BAnimation)}

The formalism / tool to be used. Currently two values and formalism are supported: \textit{BAnimation} for creating visualizations of Event-B or Classical-B models and \textit{CSPAnimation} for creating visualizations of CSP-M models.

\subsubsection{autoOpen (Type: Array, Default value: ['Events'])}

Specify the ProB views which should be opened automatically when starting the visualization. 
The following views are available: \textit{CurrentTrace}, \textit{Events}, \textit{StateInspector}, \textit{CurrentAnimations}, \textit{GroovyConsoleSession}, \textit{ModelCheckingUI}.