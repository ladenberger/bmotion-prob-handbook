\chapter{Reference}
\label{reference}

\section{The BMotion Studio API}
\label{sec:bmsapi}

(tbd)

\section{Observers}
\label{sec:observers}

Observers are used to link visual elements with the model. 
An observer is notified whenever a model has changed its state, i.e. whenever an event has been executed. 
In response, the observer will query the model's state and triggers actions on the linked visual elements in respect to the new state. 
In general, observers are defined in the Groovy script file.
\bms~comes with some predefined observers that are described in the following sections.

\subsection{Transform Observer}
\label{sec:transform_observer}

The transform observer supports changing attributes of visual elements based on the current state of the animation.
The following code snippet demonstrates the basic use of a transform observer:

\begin{lstlisting}[float=ht,language=Groovy]
transform("#myvisualelement") {
    set "fill", "green"
    set "stroke", "red"
    register(bms)
}
\end{lstlisting}

In line 1 we define a jQuery selector to select the visual elements which should be transformed.
In this case we select a visual element with the id \textit{myvisualelement} (in jQuery the prefix ``\#'' denotes that we want to select and element by its id).
jQuery provides several possibilities to select visual elements.

\info{The jQuery selector API documentation\footnote{\url{http://api.jquery.com/category/selectors/}} provides an overview and a detailed documentation about selectors.}

Line 2 to 3 are actions that are made on the visual elements which are matched by the defined selector.
In this case the \textit{fill} attribute is set to the value \textit{green} and the \textit{stroke} attribute to \textit{red}.

In line 4 we register the observer to the current visualisation.
A registered observer is triggered after every state change, e.g. after executing an event.

\subsubsection{The Use of Groovy Closures}

As we use the Groovy Scripting language for defining the observers, we can make use of the entire function and feature range of it.
Transform observers supports the use of closures for defining the selector or the value of an action.

The following code snippet demonstrates the use of closures for transform observers:

\begin{lstlisting}[float=ht,language=Groovy]
transform("#myvisualelement") {
    set "fill", { (bms.eval("1 < x").value == "TRUE") ? "white" : "lightgray" }
    set "y", {
        switch (bms.eval("x").value) {
            case "0": "15"
                break
            case "1": "20"
                break
            case "2": "25"
                break
            default: "0"
        }
    }
    register(bms)
}
\end{lstlisting}

In Groovy a closure is encapsulated in curly brackets.
Line 2 and 3 show two examples for using closures for defining the value of the attribute \textit{fill} and \textit{y} respectively.
Closures are evaluated after every state change in consequence of triggering the transform observer.
As an example, in line 2 the closure evaluates the expression $1 < x$.
The value of the \textit{fill} attribute is set to \textit{white} whenever the expression evaluates to \textit{TRUE} and otherwise to \textit{lightgray}.


\subsection{Method Observer}
\label{sec:method_observer}

\subsection{Custom Observer}
\label{sec:custom_observers}