\section{Reference}
\label{reference}

\subsection{\texttt{bmotion.json} Manifest}
\label{sec:manifest}

The options (and defaults) are:

\paragraph{name (Type: String, Default value: MyVisualization)}
\mbox{}\\\\
The name of the visualization.

\paragraph{template (Type: String, Default value: template.html)}
\mbox{}\\\\
The relative path to the HTML visualization file  (e.g. ``template.html'').

\paragraph{model (Type: String, \textit{Required})}
\mbox{}\\\\
The relative path to the model file (e.g. ``model/mymodel.mch'').

\paragraph{tool (Type: String, Default value: BAnimation)}
\mbox{}\\\\
The formalism / tool to be used. Currently two values and formalism are supported: \textit{BAnimation} for creating visualizations of Event-B or Classical-B models and \textit{CSPAnimation} for creating visualizations of CSP-M models.

\paragraph{autoOpen (Type: Array, Default value: ['Events'])}
\mbox{}\\\\
Specify the ProB views which should be opened automatically when starting the visualization. 
The following views are available: \textit{CurrentTrace}, \textit{Events}, \textit{StateInspector}, \textit{CurrentAnimations}, \textit{GroovyConsoleSession}, \textit{ModelCheckingUI}.