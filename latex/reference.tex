\section{Reference}
\label{reference}

\subsection{\texttt{bmotion.json} Manifest}
\label{sec:manifest}

The \texttt{bmotion.json} file is the root file of every \bms\ visualization.
It contains the configuration formatted using JSON (JavaScript Object Notation)\footnote{\url{http://www.json.org}.}.
The following options (and defaults) are available:

\begin{itemize}
	\item[] \textbf{name (Type: String, Default value: MyVisualization):}\\The name of the visualization. 
	\item[] \textbf{template (Type: String, Default value: template.html):}\\The relative path to the HTML visualization file  (e.g. ``template.html'').
	\item[] \textbf{model (Type: String, \textit{Required}):}\\The relative path to the model file (e.g. ``model/mymodel.mch'').
	\item[] \textbf{tool (Type: String, Default value: BAnimation):}\\The formalism / tool to be used. Currently two values and formalism are supported: \textit{BAnimation} for creating visualizations of Event-B or Classical-B models and \textit{CSPAnimation} for creating visualizations of CSP-M models.
	\item[] \textbf{autoOpen (Type: Array, Default value: ['Events']):}\\Specify the ProB views which should be opened automatically when starting the visualization. 
The following views are available: \textit{CurrentTrace}, \textit{Events}, \textit{StateInspector}, \textit{CurrentAnimations}, \textit{GroovyConsoleSession}, \textit{ModelCheckingUI}.
\end{itemize}