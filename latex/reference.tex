\section{Reference}
\label{reference}

\subsection{\bms\ Manifest File}
\label{sec:manifest}

The \bms\ manifest file (\texttt{bmotion.json}) is the root file of every \bms\ visualization.
It contains the configuration for the visualization formatted using JSON (JavaScript Object Notation)\footnote{\url{http://www.json.org}.}.
%Hence, the manifest filename must end with \texttt{.json} (denoting a JSON file).
The following options (and defaults) are available:

\begin{itemize}
	\item[] \textbf{id [Type: String, \textit{Required}]:}\\The unique id of the visualization.
	\item[] \textbf{name [Type: String]:}\\The name of the visualization.
	\item[] \textbf{model [Type: String, \textit{Required}]:}\\The relative path to the model file (e.g. ``model/mymodel.mch'').
	\item[] \textbf{template [Type: String, \textit{Required}]:}\\The relative path to the root HTML file (e.g. ``index.html'').
	\item[] \textbf{autoOpen [Type: Array, Default value: ['Events']]:}\\Specify the ProB views which should be opened automatically in the respective view when starting the visualization. 
The following views are available: \textit{CurrentTrace}, \textit{Events}, \textit{StateInspector}, \textit{CurrentAnimations}, \textit{GroovyConsoleSession}, \textit{ModelCheckingUI}.
	\item[] \textbf{views [Type: list]:}\\ List of view objects that are opened in separate windows.
	At least one view must exists.
	A view object has the following options:
	\begin{itemize}
		\item[] \textbf{id [Type: String, \textit{Required}]:}\\ Unique id of the view.
		\item[] \textbf{template [Type: String, \textit{Required}]:}\\ The relative path to the HTML visualization file (e.g. ``template.html'').
		\item[] \textbf{name [Type: String]:}\\ The name of the view.
		\item[] \textbf{width [Type: Integer]:}\\ The width of the view.
		\item[] \textbf{height [Type: Integer]:}\\ The height of the view.
	\end{itemize}
	%\item[] \textbf{tool [Type: String, Default value: BAnimation]:}\\The formalism / tool to be used. Currently two values and formalism are supported: \textit{BAnimation} for creating visualizations of Event-B or Classical-B models and \textit{CSPAnimation} for creating visualizations of CSP-M models.
\end{itemize}

Example \bms\ manifest file (e.g. \texttt{lift.json}):

\begin{lstlisting}[language=JavaScript]
{
  "id": "lift",
  "name": "Simple lift",
  "template": "lift.html",
  "model": "model/lift.mch",
  "autoOpen": [
  	"CurrentTrace",
    "Events"
  ],
  "views": [
    {
      "id": "buttons",
      "name": "Lift control",
      "template": "control.html",
      "width": 300,
      "height": 200
    }
  ]
}
\end{lstlisting}

\subsection{\texttt{bmotion.json} Configuration File}
\label{sec:configfile}

The \texttt{bmotion.json} config file is located in the \texttt{resources} folder of your \bms\ installation.
It contains the configuration for the \bms\ application formatted using JSON (JavaScript Object Notation)\footnote{\url{http://www.json.org}.}.

\warning{Please note: Normally, the user should not edit this file!}

The following options (and defaults) are available:

\begin{itemize}

	\item[] \textbf{socket [Type: object]:}\\ Settings for websocket server. The following options are available:
	\begin{itemize}
		\item[] \textbf{host [Type: String, Default value: localhost]:}\\ The host for the websocket server.
		\item[] \textbf{port [Type: String, Default value: 19090]:}\\ The port for the websocket server.
	\end{itemize}

	\item[] \textbf{prob [Type: object]:}\\ Settings for websocket server. The following options are available:
	\begin{itemize}
		\item[] \textbf{host [Type: String, Default value: localhost]:}\\ The host for the ProB2 server.
		\item[] \textbf{binary [Type: String, Default value: ./cli/]:}\\ Path to ProB binary.
	\end{itemize}

\end{itemize}