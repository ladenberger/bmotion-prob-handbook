\section{Reference}
\label{reference}

\subsection{\bms\ Manifest File}
\label{sec:manifest}

A visualization template is identified by a \textit{manifest file}.
The manifest file is the root file of every interactive formal prototype.
It contains the configuration for the interactive formal prototype formatted using JSON (JavaScript Object Notation).\footnote{http://www.json.org/}
\Cref{tab:manifest_options} gives an overview of the available options.
The table shows the option's name, its type, a short description and denotes if the option is required or optional.
\Cref{lst:lift_manifest} exemplifies the use of a manifest file based on the interactive formal prototype of the \eventB\ simple lift system.
	
\begin{table}
\centering\footnotesize
\caption{Available options for \bmsweb\ manifest file}
\begin{tabular}{lllp{9.8cm}}
\toprule[2pt]
\textbf{Name} & \textbf{Type} & \textbf{Required} & \textbf{Description} \\
\midrule[2pt]
\textit{id} & string & yes & Unique id of the interactive formal prototype\\
\midrule[2pt]
\textit{name} & string & no & The name of the interactive formal prototype\\
\midrule[2pt]
\textit{template} & string & yes & The relative path to the HTML template file (e.g. ``template.html'').\\
\midrule[2pt]
\textit{model} & string & yes & The relative path to the formal specification file that should be animated (e.g. ``model/mymodel.mch'') \\
\midrule[2pt]
\textit{modelOptions} & map & no & A key/value map defining the options for loading the model - The available options are dependent on the animator and formalism \\
\midrule[2pt]
\textit{autoOpen} & array & no & The user can specify the \prob\ views which should be opened automatically when running the interactive formal prototype - The following views are available for \prob\ animations: \textit{CurrentTrace}, \textit{Events}, \textit{StateInspector} and \textit{ModelCheckingUI} \\
\midrule[2pt]
\textit{views} & list & no & List of additional views - A view object has the following options: \\
\midrule[0.5pt]
\hspace{0.3cm} \textit{id} & string & yes & Unique id of the view \\
\midrule[0.5pt]
\hspace{0.3cm} \textit{name} & string & no & The name of the view \\
\midrule[0.5pt]
\hspace{0.3cm} \textit{template} & string & yes & The relative path to the HTML template file of the view (e.g. ``view1.html'') \\
\midrule[0.5pt]
\hspace{0.3cm} \textit{width} & numeric & no & The width of the view \\
\midrule[0.5pt]
\hspace{0.3cm} \textit{height} & numeric & no & The height of the view \\
\bottomrule[2pt]
\end{tabular}
\label{tab:manifest_options}
\end{table}	

\begin{minipage}{\linewidth}
\begin{lstlisting}[language=JavaScript, caption={Example manifest file for the simple lift system (JSON)}, label={lst:lift_manifest}]
{
  "id": "lift",
  "name": "Simple lift system",
  "template": "lift.html",
  "model": "model/m2.bcm",
  "autoOpen": [
    "CurrentTrace",
    "Events"
  ]
}
\end{lstlisting}
\end{minipage}

\subsection{\texttt{bmotion.json} Configuration File}
\label{sec:configfile}

The \texttt{bmotion.json} config file is located in the \texttt{resources} folder of your \bms\ installation.
It contains the configuration for the \bms\ application formatted using JSON (JavaScript Object Notation)\footnote{\url{http://www.json.org}.}.

\warning{Please note: Normally, the user should not edit this file!}

The following options (and defaults) are available:

\begin{itemize}

	\item[] \textbf{socket [Type: object]:}\\ Settings for websocket server. The following options are available:
	\begin{itemize}
		\item[] \textbf{host [Type: String, Default value: localhost]:}\\ The host for the websocket server.
		\item[] \textbf{port [Type: String, Default value: 19090]:}\\ The port for the websocket server.
	\end{itemize}

	\item[] \textbf{prob [Type: object]:}\\ Settings for websocket server. The following options are available:
	\begin{itemize}
		\item[] \textbf{host [Type: String, Default value: localhost]:}\\ The host for the ProB2 server.
		\item[] \textbf{binary [Type: String, Default value: ./cli/]:}\\ Path to ProB binary.
	\end{itemize}

\end{itemize}
