
\section{Outline}

\begin{enumerate}
\item The Rodin Platform (\ref{reference_01}): This chapter should give the user an overview
  about the UI elements he might encounter.
  \begin{enumerate}
  \item Eclipse in general: just references to other resources
  \item The Event-B perspective: We briefly describe the various views
    (Event-B explorer, structure editor, outline view, symbol view)
    and their content.
    We want to cover every element of each view (like buttons, entry fields, etc.),
    but for the underlying theory we link to the other reference sections.
    The relevant menu entries are also described.
  \item The proving perspective: We do basically the same as for the Event-B perspective.
    The views are proof tree, goal, control and statistics.
  \item Preferences: We briefly describe what preference can be set for Rodin.
    For their deeper meaning, we'll refer to the other sections (esp. the prover settings).
  \end{enumerate}
\item Event-B (modeling notation) (\ref{reference_02}):
  \begin{enumerate}
  \item Contexts
    \begin{enumerate}
    \item Sets: Introducing sets, with a comment about how to add enumerated sets
    \item Constants and axioms (with theorems), which POs are generated
    \item Extending a context
    \end{enumerate}
  \item Models
    \begin{enumerate}
    \item Variables and invariants (with theorems), generated POs 
    \item Seeing a context
    \item Refining a model: Refines clause, gluing invariant, general picture of refinement with
      pointers to the event below, re-use of variables
    \item Events: All aspects of events are covered: Parameters, guards and actions,
      generated POs, refinement (again with generated POs, re-use of parameters, witnesses),
      status (normal/convergent/anticipated), merging of events.
    \item Termination: How to prove termination by the model's variant and the status of events
    \end{enumerate}
  \item Generated proof obligations: We give a brief overview about what POs are generated
    where. This should help the user to identify the reason of a PO when he just know its label.
  \end{enumerate}
\item Mathematical notation (\ref{reference_03}):
  Here we cover the complete mathematical notation of Event-B. For each expression
  (like an operator) we describe its purpose, its type, the type of its arguments and
  well-definedness-conditions. 
  We roughly separate the expressions into three groups: predicates, set-theoretical and arithmetic.

  Some laws (like the commutative law of addition in arithmetic) would be nice but will not be part of the first
  iteration of the documentation. E.g. the Z reference manual does this.

  References to related proof rules would be nice to have, too. But again, this is not part of the first iteration.
  \begin{enumerate}
  \item Introduction: What data types exist, what are well-definedness-conditions, 
    how the description of the expressions is organized.
  \item Predicates: All operators that work with predicates ($\land$, $\lor$, quantifier, \ldots). 
  \item Sets and relations
  \item Arithmetic
  \end{enumerate}
\item Proving (\ref{reference_04}): This section should enable the user to understand proofs in Event-B.
  In the first iteration, we won't go into detail when it comes to proof rules and tactics. We
  just refer to the already existing Wiki sites.
  \begin{enumerate}
  \item Introduction: How does a proof in Event-B looks like? (Goal, proof tree, hypothesis, selected hypothesis, \ldots)
  \item Proof rules (first only a link to the Wiki page)
  \item Proof tactics (first only a link to the Wiki page)
  \item The provers (PP, ML, New-PP): We just give a very brief over the existing provers.
    (What is its strength/weakness).
  \end{enumerate}
\item Glossary (\ref{reference_05}):
  During planning the outline of the reference we encountered some
  topics that did not seem to fit anywhere. This might be a good place
  to put such things.
  (E.g. horizontal vs. vertical refinement, naming convention, \ldots)
\item Index (optional) (\ref{reference_06}):
  An index is very helpful to find the relevant information in the documentation.
  We make the index optional because there is the alternative of a text-based search (esp. in the
  web version of the documentation) and we ran into technical difficulties when generating the 
  index for the web version.
\end{enumerate}

%%% Local Variables: 
%%% mode: latex
%%% TeX-master: "rodin-doc"
%%% End: 
