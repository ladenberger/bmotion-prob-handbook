\documentclass[12pt]{book}
\usepackage[left=2.5cm,top=3cm,right=2.5cm,bottom=3cm]{geometry}
\usepackage[hyphens]{url} 
\usepackage[colorlinks=true,
	        pdftex,
            plainpages=false,
            pdfauthor={Lukas Ladenberger (Ed.)},
            pdftitle={BMotion User's Handbook},
            pdfsubject={},
            pdfkeywords={BMotion Studio, ProB, Classical-B, Event-B},
            pdfproducer={http://www.stups.hhu.de/ProB},
            pdfcreator={plastex-based tool chain}]{hyperref}
\usepackage{graphicx}
\usepackage{bsymb}
\usepackage{b2latex}
\usepackage{fancyhdr,lastpage,color}
\usepackage{verbatim}
\usepackage{wrapfig}
\usepackage{makeidx}
\usepackage{fix-cm}
\usepackage[utf8]{inputenc}

\usepackage{listings}

\usepackage{longtable}

\lstdefinelanguage{Groovy}% 
{morekeywords={abstract,any,as,boolean,break,byte,case,catch,char,class, 
const,continue,def,default,do,double,else,extends,false,final,finally, 
float,for,goto,if,implements,import,instanceof,in,int,interface,label, 
long,native,new,null,package,private,protected,public,return,short, 
static,strictfp,super,switch,synchronized,this,throw,throws,transient, 
true,try,void,volatile,while,with},% 
sensitive=true,% 
morecomment=[l]//,% 
morecomment=[s]{/}{/},% 
morestring=[b]",% 
morestring=[b]',% 
}[keywords,comments,strings]

\lstdefinelanguage{B}% 
{morekeywords={MACHINE, REFINEMENT,OPERATIONS,INITIALISATION,END,BEGIN,PRE,IF,THEN,ELSE,VARIABLES,INVARIANT,SETS},% 
sensitive=true,% 
morecomment=[l]//,% 
morecomment=[s]{/}{/},% 
morestring=[b]",% 
morestring=[b]',% 
}[keywords,comments,strings]

\lstset{
  language={Java},basicstyle=\ttfamily\footnotesize,,columns=fullflexible
}
\lstset{
  language={Groovy},basicstyle=\ttfamily\footnotesize,,columns=fullflexible
}
\lstset{
 numberbychapter=false
}

\definecolor{oldmauve}{rgb}{0.4, 0.19, 0.28}

\lstset{
  frame=top,frame=bottom,
  %basicstyle=\small\normalfont\sffamily,    % the size of the fonts that are used for the code
  stepnumber=1,                           % the step between two line-numbers. If it is 1 each line will be numbered
  numbersep=10pt,                         % how far the line-numbers are from the code
  tabsize=1,                              % tab size in blank spaces
  extendedchars=true,                     %
  breaklines=true,                        % sets automatic line breaking
  captionpos=t,                           % sets the caption-position to top
  mathescape=false,
  stringstyle=\color{blue}\ttfamily, % Farbe der String
  keywordstyle=\color{oldmauve}\ttfamily, % Farbe der String
  showspaces=false,           % Leerzeichen anzeigen ?
  showtabs=false,             % Tabs anzeigen ?
  xleftmargin=17pt,
  framexleftmargin=17pt,
  framexrightmargin=17pt,
  framexbottommargin=5pt,
  framextopmargin=5pt,
  showstringspaces=false,      % Leerzeichen in Strings anzeigen ?
  escapechar=\%
 }

\definecolor{lightgray}{rgb}{.9,.9,.9}
\definecolor{darkgray}{rgb}{.4,.4,.4}
\definecolor{purple}{rgb}{0.65, 0.12, 0.82}

\lstdefinelanguage{JavaScript}{
  keywords={typeof, new, true, false, catch, function, return, null, catch, switch, var, if, in, while, do, else, case, break},
  keywordstyle=\color{blue}\bfseries,
  ndkeywords={class, export, boolean, throw, implements, import, this},
  ndkeywordstyle=\color{darkgray}\bfseries,
  identifierstyle=\color{black},
  sensitive=false,
  comment=[l]{//},
  morecomment=[s]{/*}{*/},
  commentstyle=\color{purple}\ttfamily,
  stringstyle=\color{red}\ttfamily,
  morestring=[b]',
  morestring=[b]"
}

\newcommand{\bms}{BMotion Studio}

% Rodin Handbook Version Path. "current" is the newest version of the handbook. This should be changed if we want to build a handbook for another (i.e. older version)
\newcommand{\versionpath}{nightly}

% Absolute path to handbook
\newcommand{\handbookpath}{http://nightly.cobra.cs.uni-duesseldorf.de/bmotion/bmotion-prob-handbook}

\newcommand{\bms}{BMotion Studio}

% We generate an index
\makeindex


% defining a if(plastex) environment
\newif\ifplastex
\plastexfalse

\def\doculist#1#2{
\begin{quote}
\hspace{-10mm}
\textrm{\includegraphics[width=7mm]{#2}} % Hack!  We "mark" the image with textrm so that we can use a different CSS-Style in plastex.
\vspace{-8mm}

#1
\end{quote}
}

\def\tick#1{\doculist{#1}{img/tick_64.png}}
\def\info#1{\doculist{#1}{img/info_64.png}}
\def\warning#1{\doculist{#1}{img/warning_64.png}}
\def\pencil#1{\doculist{#1}{img/pencil_64.png}}

% macro for icons
\def\icon#1{
\includegraphics[]{img/icons/#1}
}

% macro for image versions (pdf version + html version)
% #1 Path to image for pdf version
% #2 Path to image for html version
% #3 Caption
% #4 Label
\def\imagedpi#1#2#3#4#5{
	\ifplastex
		\begin{figure}[!h]
		\begin{center}
			\includegraphics{#3}
			\caption{#4}
			\label{#5}
		\end{center}
		\end{figure}
	\else
		\begin{figure}[!h]
		\begin{center}
			\includegraphics[width=#2]{#1}
			\caption{#4}
			\label{#5}
		\end{center}
		\end{figure}
	\fi
}

% different method to write an ASCII backslash for plastex and normal pdflatex
\ifplastex
  \newcommand{\mybackslash}{\textbackslash}
\else
  % we do not use textbackslash for latex, because it does not use the current font setting
  \newcommand{\mybackslash}{\symbol{`\\}}
\fi

% Path to resources like zip's with machines
% We use a relative path in the html + eclipe version (in order to work offline)
% and an absolute path in the pdf version
\ifplastex
	\newcommand{\filepath}{files/}
\else
	\newcommand{\filepath}{\handbookpath/\versionpath/files/}
\fi

% Use this definition to create a link to the file. The definition takes to arguments. 
% The first argument (1) defines the file name i.e. Celebrity.zip or in case if you saved 
% the file in a subdirectory subdirecotry/Celebrity.zip. The second argument (2) defines 
% the name which should be displayed in the document, i.e. Celebrity Problem Example Download
\def\file#1#2{
\href{\filepath#1}{#2}
}

% We want to mark contributions from other plugins in a special way, by including the plugin's
% icon and by putting the content in a gray box.  We have to approach this differently for
% Latex and for Plastex:
% Latex: We use "shaded" from package "framed"
% Platexte: We use "verse" as the marker and create the shading with the style sheet.
\newcommand{\tmpName}{Dummy}
\ifplastex
\newenvironment{rodin-plugin}[2]
{
\renewcommand{\tmpName}{#2}
  \begin{verse}
\begin{wrapfigure}{l}{}
    \includegraphics{#1}
\end{wrapfigure}
}
{
\newline
\textit{This contribution requires the \textbf{\tmpName} plugin.  The content is maintained by the plugin contributors and may be out of date.}
\end{verse}
}
\else
\usepackage{framed}
\definecolor{shadecolor}{rgb}{0.93,0.93,0.93}
\newenvironment{rodin-plugin}[2]
{
\renewcommand{\tmpName}{#2} % Otherwise we cannot use #2 in the end block - stupid!
\begin{shaded}
\begin{wrapfigure}{l}{10mm}
\vspace{-5mm}
\includegraphics[width=10mm]{#1}
\vspace{-5mm}
\end{wrapfigure}
\noindent
}
{
\vspace{1mm}
\noindent\rule{\textwidth}{.1pt}
\vspace{1mm}
\noindent
{\scriptsize This contribution requires the \textbf{\tmpName} plugin.  The content is maintained by the plugin contributors and may be out of date.}

\end{shaded}
}
\fi

% Marginpars are  cropped - this formats them nicely.
\let\oldmarginpar\marginpar
\renewcommand\marginpar[1]{\-\oldmarginpar[\raggedleft\scriptsize{#1}]
{\raggedright\small{#1}}}
\marginparwidth=2cm

% A command to typeset names of an Event-B section (like variables, invariant, etc)
% consistently.
\newcommand{\eventbsection}[1]{\textsl{#1}}

% A command to typeset consistently the names of proof obligations
\newcommand{\eventbpo}[1]{\textsf{#1}}

% Commands for the structure of the reference section

% rrnames is used for the array of operator symbols and description
% at the beginning of a reference section.
% The environment defines an array with three columns:
% 1) The mathematical symbol
% 2) The ASCII representation
% 3) A description of the operator
\newenvironment{rrnames}%
  {\begin{tabular}{l@{\quad---\quad}l@{\quad---\quad}l}}%
    {\end{tabular}}

% The environment rodinrefentry is used for a reference section with several
% entries: Description, Definition, Types, Well-Definedness
% \rrindent is the indention in such an environment
\newlength{\rrindent}
\setlength{\rrindent}{8em}
\newenvironment{rodinrefentry}{%
   \renewcommand\descriptionlabel[1]{\makebox[\rrindent][r]{\textbf{##1}}}
   \setlength{\leftmargini}{\rrindent}
   \begin{description}%
}{%
   \end{description}%
}
\newcommand{\rrdesc}{\item[Description]}
\newcommand{\rrdef}{\item[Definition]}
\newcommand{\rrtypes}{\item[Types]}
%\newcommand{\rrwd}{\item[Well-Definedness]}
\newcommand{\rrwd}{\item[WD]}
\newcommand{\rrfis}{\item[Feasibility]}

\newcommand{\actfis}{\mathcal{F}}

% operators (L and D) for well-definendness
\newcommand{\wdl}{\mathcal{L}}
\newcommand{\wdd}{\mathcal{D}}

% a placeholder symbol for operators
\newcommand{\opelipse}{\mathbin{\Box}}

\newcommand{\podef}[4]{%
  \begin{center}
    \setlength{\parindent}{2em}\vspace{0.2em}
    \begin{tabular}{rp{0.5\textwidth}}
      \hline
      & \textbf{#1} \\
      Name       & #2 \\
      Goal       & #3 \\
      Hypotheses & #4 \\
      \hline
    \end{tabular}
  \end{center}
}

% a second approach to proof obligations
\newcommand{\pode}[3]{%
  \begin{center}
    \setlength{\parindent}{2em}\vspace{0.2em}
    \begin{tabular}{rp{0.6\textwidth}}
      \hline
      & \textbf{#1} \\
      Name       & #2 \\
      Goal       & #3 \\
      \hline
    \end{tabular}
  \end{center}
}

%%% Local Variables: 
%%% mode: latex
%%% TeX-master: "rodin-doc"
%%% End: 


\title{BMotion Studio for ProB}
\author{
Work in Progress\\
Handbook $ $Rev: 13274 $ $ \\
\\
\href{mailto:ladenberger@cs.uni-duesseldorf.de}{ladenberger@cs.uni-duesseldorf.de}\\
\href{http://www.stups.hhu.de/ProB}{http://www.stups.hhu.de/ProB}
}

\begin{document}        
\pagestyle{empty}
\ifplastex
\maketitle
\else
\pagenumbering{roman}
\begin{titlepage}
\AddToShipoutPicture*{\BackgroundPic}
\vspace*{\titletop}
{\titledimrodin\selectfont \bfseries BMotion Studio}

\vspace*{\titlesubtitledistance}
{\titledimhandbook\selectfont \bfseries User's Handbook}

\vspace*{\titlesecdistance}
{\titledimsubtext\selectfont \textbf{\textsf{ProB Edition}}}

\vspace*{\titlesecdistance}
{\titledimsubtext\selectfont \textsf{Lukas Ladenberger (Editor)}}

\vspace*{\titlesecdistance}
{\titledimsponsor\selectfont %
  $\hbox{\includegraphics[height=4ex]{img/advance-logo.png}}$
  \textsf{This work is sponsored by the Advance Project}}

\vspace*{\titlebottom}

\end{titlepage}
\ifdefined\isinprint
\cleardoublepage
\clearpage
\else\fi
\cleardoublepage
\pagenumbering{arabic}
\phantomsection
\addcontentsline{toc}{chapter}{Contents}
\tableofcontents
\fi
\newpage

\section{Overview}

This handbook consists of three parts:

\begin{description}
	\item[Introduction] You are reading the introduction right now.  It helps you to orient yourself and to find information quickly.
	\item[Tutorial] If you are completely new to BMotion Studio, the Tutorial is a good way to get up to speed quickly.  It guides you through installation and usage of the tool and gives you an overview of the BMotion Studio features.
	\item[Reference] The reference section provides comprehensive documentation of BMotion Studio, and its components.
	\item[Frequently Asked Questions] Common issues are listed by category in the FAQ.
	\item[Index] Particularly for the print version of this handbook, we included an index.  In the electronic versions, you may want to try the search functionality as well.
\end{description}

\subsection{Formats of this Handbook}

The handbook comes in various formats:

\begin{description}
	\item[Eclipse Help] BMotion Studio is shipped with Eclipse and can be accessed through the help system.
	%\item[Online Help] You can access the handbook online at \url{http://handbook.event-b.org}.
	\item[PDF Help] The BMotion Studio Website \url{http://www.stups.hhu.de/ProB/index.php5/BMotion_Studio} includes a link to the PDF version of the handbook.
\end{description}

\subsection{Background and Motivation}

Formal methods like the B-Method gained a lot of popularity as approach for the specification and design of software that ensures its safety and reliability. Several industrial projects like the Coppilot System that controls the opening and closing of platform screen doors at the paris metro line 13 have been installed successfully. Such industrial applications need also industrial strength tools in order to support the deployment of formal models. One of them is ProB. ProB is an animation tool with the challange to check the presence of desired functionality and to inspect the behavior of a formal model.

Another challenge for successful deployment of formal models is the communication between a developer and a domain expert (or manager). On the one hand it is crucial for the developer to get feedback from the domain expert for further development. On the other hand the domain expert needs to check whether his expectations are met. Furthermore, good communication and presentation of formal models can help to obtain several contracts. An animation tool like ProB could perform this task, but may be still too difficult for domain experts, because they also need a certain level of knowledge about the mathematical notation to understand the meaning of a specific state. To avoid this problem, it is useful to create domain specific visualizations.

That's why we introduce BMotion Studio for ProB, a tool built on top of the ProB Java API for creating domain specific visualisations of Classical-B, Event-B and CSP models.

\subsection{BMotion Studio Website}
\label{rodin_wiki}

This handbook is complemented by the BMotion Studio for ProB Website (\url{http://www.stups.hhu.de/ProB/index.php5/BMotion_Studio}).  Sometimes, the handbook will refer to the website for more information.

\subsection{Feedback}
\label{feedback}

All online versions of the handbook contain a button or link for feedback.  You can also submit feedback via email to \texttt{ladenberger@cs.uni-duesseldorf.de}.

\section{Conventions}
\label{conventions}

We use the following conventions in this manual:

\tick{Checklists and Milestones are designated with tick. Here we summarize what we want to learn or should have learned so far.}
\info{Useful information and tricks are designated by the information sign.}
\warning{Potential problems and warnings are designated by a warning sign.}
\pencil{Examples and Code are designated by a pencil.}

We use \texttt{typewriter} font for file names and directories.

We use \textsf{sans serif font} for GUI elements like menus and buttons.  Menu actions are depicted by a chain of elements, separated by ``$\rangle$'', e.g. \textsf{File $\rangle$ New $\rangle$ Project...}.

\section{Acknowledgements}
\label{sec:acknowledgements}

The icons that you find throughout this handbook were created by Pixel-Mixer\footnote{\url{http://pixel-mixer.com/}}, who provides them freely.  Thanks!

\section{ADVANCE}
\label{advance}

This work has been sponsored by the Advance project\footnote{\url{http://www.advance-ict.eu/}}.  ADVANCE is an FP7 Information and Communication Technologies Project funded by the European Commission. The overall objective of ADVANCE is the development of a unified tool-based framework for automated formal verification and simulation-based validation of cyber-physical systems.

The ADVANCE project is unique in addressing both simulation and formal verification within a single design framework.

Unification is being achieved through the use of a common formal modelling language supported by methods and tools for simulation and formal verification. An integrated tool environment is providing support for construction, verification and simulation of models.

ADVANCE is building on an existing formal modelling language - Event-B - and its associated tools environment - Rodin - with strong support for formal verification. In ADVANCE, Rodin is being further strengthened and augmented with novel approaches to multi-simulation and testing.

\section{Creative Commons Legal Code}
\label{sec:cc}        

The work presented here is the result of an collaborative effort
that took many years.  To ensure that access to this work stays free
and to avoid any legal ambiguities, we decided to formally lincense
it under the Creative Commons Share-Alike License.

This work is licensed under the Creative Commons Attribution-ShareAlike 3.0 Unported License. To view a copy of this license, visit \url{http://creativecommons.org/licenses/by-sa/3.0/} or send a letter to Creative Commons, 444 Castro Street, Suite 900, Mountain View, California, 94041, USA.



% \section{Style Guide}
\label{style-guide}

\info{For now, we will manage the style guide as a \LaTeX~document together with the rest of the documentation.  We may take it out upon publication.}

\subsubsection{General Stylistic Guidelines}

\begin{itemize}
	\item The conventions (\ref{conventions}) are part of the style guide.

	\item Use the ``we'' form.

	\item We use British English.

	\item When referring to the different views in Rodin, the word ``view'' should be written in lowercase, e.g., the Rodin Problems view.
\end{itemize}

\subsubsection{Files}

Files should be saved in the \texttt{files} subdirectory. You can also create a subdirectory. Then, use the definition 

\begin{verbatim} \file{1}{2} 
\end{verbatim} 

to create a link to the file. The definition extends to arguments. The first argument (1) defines the file name i.e. \texttt{Celebrity.zip} or if you saved the file in a subdirectory \texttt{subdirectory/Celebrity.zip}. The second argument (2) defines the name which should be displayed in the document, i.e. ``Celebrity Problem Example Download''.

\warning{Please note, that you only enter the file name without a path before (expected subdirectories). The build script assigns the correct path to the file on the server automatically .}

Here is an example using the definition: \file{Celebrity.zip}{Celebrity Problem Example Download}.

\subsubsection{Avoiding Redundancy}

We will reduce (or avoid) redundancy through heavy linking, following these guidelines:

\begin{itemize}
	\item If in doubt, provide the bulk of the information in the Reference section.  For instance, the FAQ entry ``What is Event-B?''  Should simply refer to the Event-B entry in the Reference section.
	\item Web Links should not appear multiple times
	\item List web links as footnotes in the Tutorial and FAQ.
	\item List web links in a ``See also'' Section in the Reference.
\end{itemize}

\subsubsection{Sections}

\begin{itemize}
	\item If referring to a specific chapter or section, use uppercase to denote it, e.g. ``in Chapter~3''.
	\item We also refer to subsections as Section, e.g. ``see Section~2.5''
	\item We have a small number of well-defined chapters which are the top level structuring element.
	\item Sections and subsections are numbered.  In the HTML-Versions, they are broken into subpages.
    \item Subsubsections do not receive numbers and are not broken into subpages in the HTML.  Keep this in mind regarding both the reading flow and page sizes.
	\item Avoid linking (ref) to subsubsections, as they don't have a number.  Latex will instead provide a link to the next higher element.  This works, but it could create confusion.
	\item Generally, we should avoid gaps in the hierarchy (i.e. having a subsubsection in a section without a subsection in between).\footnote{Coincidentally, this style guide violates this rule. Reason: We want the style guide to remain whole and not be broken into subsections, but the proper hierarchy is a section.}
	\item Section labels should be all in lower case. Use ``\_'' for blanks.
	\item We use the prefix ``\texttt{int\_}'' for introduction section labels, ``\texttt{tut\_}'' for tutorial section labels following a meaningful description of the section (i.e. \texttt{tut\_rodin\_installation}) and ``\texttt{faq\_}'' for faq section labels following a short version of the title (i.e. \texttt{faq\_diff\_eventb\_b}). Reference section labels have no prefix.
\end{itemize}

\subsubsection{Images}
\begin{itemize}
	\item Images must be no more than 700 pixels in width (for HTML version) and no more than 160 mm (for PDF version). This is fairly easy for bitmaps (screenshot), pay attention to this regarding how plasTeX converts vector images. (see Latex section below on how to include images)

	\item Screenshots should look neat and consistent.  Horizontal real estate will always be an issue, so please resize the windows before taking the screenshot to keep things readable at 700 pixel width.  (see Latex section below on how to include images)

	\item Images should always have a caption and a label. Please use the same conventions for the figure labels that are used for the the section labels, but use the prefix ``\texttt{fig\_}''. \\Example: ``\texttt{fig\_tut\_03\_traffic\_light}''. For referencing do not use references like "As shown bellow" or "Like in the following image:". Use always a format like:

\begin{verbatim}"... as shown in figure \ref{fig_tut_03_traffic_light}."\end{verbatim}

	\item We use icons from Pixel-Mixer, which are free as long as credit is given: \url{http://www.softicons.com/free-icons/toolbar-icons/basic-icons-by-pixelmixer}

  \item We include Window decoration only when it is really necessary.  If we discuss only some views, we crop the rest away.  Please crop neatly, following edges. Even if you need a screenshot with window decoration, you should always use the same Window decoration (i.e. linux ubuntu default decoration style). If you need such a screenshot, please contact Lukas.

  \item Image file names should be all in lower case and not include umlaute or special characters. Use ``\_'' for blanks.

  \item Image files should keep the following rules:
	
\begin{itemize}
		\item Tutorial images should be saved in the sub folder \texttt{img/tutorial} with the prefix ``\texttt{tut\_}'' following the section number. For instance, \texttt{tut\_01\_image1.png}.

		\item FAQ images should be saved in the sub folder \texttt{img/faq} with the prefix ``\texttt{faq\_}''. For instance, \texttt{faq\_image1.png}.

	\end{itemize} 
	\end{itemize}

\subsubsection{Icons}

For icons (i.e. RODIN platform icons) in continuous text use the command: 

\begin{verbatim} \icon{1} \end{verbatim} 

The first (1) argument takes the path to the icon (i.e. \texttt{rodin/auto\_prover.png}). You can find all icons in the folder \texttt{latex/img/icons/}

Here is an example using icons in continuous text:

\textit{Lorem ipsum dolor sit amet, consetetur \icon{rodin/auto_prover.png} sadipscing elitr, sed diam nonumy eirmod tempor invidunt ut labore et dolore magna aliquyam erat, sed diam voluptua. At vero eos et accusam et justo duo dolores et ea rebum. Stet clita kasd gubergren, no sea takimata sanctus est Lorem ipsum dolor sit amet. Lorem ipsum dolor sit amet, consetetur sadipscing elitr, sed diam nonumy eirmod tempor invidunt ut labore et dolore \icon{rodin/lasoo_prover.png} magna aliquyam erat, sed diam voluptua. At vero eos et accusam et justo duo dolores et ea rebum. Stet clita kasd gubergren, no sea takimata sanctus est Lorem ipsum dolor sit amet.}

\subsubsection{Index}
Please add meaningful entries to the index. This helps users to look up information more efficiently.
The \LaTeX-command to do this is \verb#\index{#\textit{word}\verb#}#.
\begin{itemize}
\item The most important thing to think about when considering the index is the user perspective: 
  What is the word that a user will look up to try and find this topic?
\item An index entry should be a noun, singular and written in lowercase (uppercase if it's a name).
\item If the word appears in are several locations in the documentation (e.g. in the tutorial and the
  reference section), try to highlight the most commonly used location with \verb#\index{word|textbf}#.
  This is usually an entry in the reference section.
\item Adjectives should be used with care. 
  Please use them only if omitting the adjective changes the meaning of the entry drastically or
  if the adjective is necessary to distinguish the entry from other entries with the same noun but
  other location.
\item If you use an adjective, consider using the noun as a key for sorting: \verb#\index{cloud,grey@grey cloud}#
  instead of \verb#\index{grey cloud}#
\item If you have multiple entries for the same word, consider using sub-entries (\verb#\index{cloud!grey}#)
  if different things are meant. Another example is \verb#\index{true!as predicate}# and \\ \verb#\index{true!as expression}#
\item Multiple index entries for the same text block are alright.
\item If you use the index command, please have a look at the result.
\end{itemize}

\subsubsection{\LaTeX{} Styling}

\begin{itemize}
	\item Try to avoid fancy \LaTeX formatting, as PlasTeX (used for generating HTML) is temperamental.  Macros are especially problematic, and sometimes the result is just ugly.
	\item We have the option to use different files to generate the PDF and HTML, but we would generally prefer not to do this.  Look at \texttt{bsymb.sty} and \texttt{plastex-bsymb.sty} as an example. \textbf{NOTE:} We don't do this any more for any style files.
	\item Every section should have a label, reflecting the section name, all lowercase, spaces replaced with underscores (\_).
	\item Don't create subdirectories in the \texttt{latex} folder, as the scripts cannot always deal with them.
	\item Put images in the \texttt{img} folder.  Feel free to create additional directory structures underneath.
	\item Files other than images (e.g. Event-B projects) --- we have macros to link them absolute in the PDF, and to link them relative in HTML and Eclipse Help.
	\item Use linked sections numbers (generated with \texttt{\\ref{}} instead of hyperlinks for cross-references. This is necessary for the print documentation to be useful.
	\item When including images in Latex, do not provide a width!  Instead, try to embed the print size in the image itself.  For instance, PNGs allow you to set the print size (in mm).  This way we can be sure that the images are rendered as HTML without distortion.
\end{itemize}

\subsubsection{Contributions from Plugin Developers}
\label{sec:plugin_contributions}

\begin{rodin-plugin}{prob.png}{ProB}
We want to encourage Plugin developers to contribute to the Handbook, but we have to make it clear that we cannot maintain that documentation.  Therefore, it has to be clearly marked.  We use a custom environment for that purpose that
(1) provides the Plugin's icon, (2) adds a disclaimer to the end of the custom documentation and (3) puts the content into a gray box (like this one).

There are some limitations to the environment that we use.  Specifically, headlines and images should be kept outside the gray box.

\end{rodin-plugin}

%%% Local Variables: 
%%% mode: latex
%%% TeX-master: "rodin-doc"
%%% End: 


\chapter{Tutorial}
\label{tutorial}


The objective of this chapter is to get you to a stage where you can use BMotion Studio to visualize formal models.  
We expect you to have a basic understanding of logic and an idea why doing formal modelling is a good idea.  
You should be able to work through the tutorial with no or little outside help.

This tutorial covers installation and configuration of BMotion Studio; it brings you through step by step through building visualizations for formal models and it provides the essential theory and provides pointers to more information.

We attempt to alternate between theory and practical application thereby keeping you motivated.  
We encourage you not to download solutions to the examples but instead to actively build them up yourself as the tutorial progresses.

If something is unclear, remember to check the Reference (\ref{reference}) for more information.

\section{Outline}

\begin{description}
	\item[Background before getting started (\ref{tutorial_01})] We give a brief description of what BMotion Studio is and what it is being used for and what kind of background knowledge we expect.
	\item[Installation and first steps (\ref{tutorial_02})] We guide you through downloading, installing and starting BMotion Studio. 
	\item[We create our first visualisation (\ref{tutorial_03})] We introduce a first visualization for a simple lift Event-B model. 
	
\end{description}



\section{Before Getting Started}
\label{tutorial_01}

Before we get started with the actual tutorial, we are going to go over the required background to make sure that you have a rudimentary understanding of the necessary concepts.

\tick{\textbf{You can skip this section, if...}
\begin{itemize}
	\item ... you a familiar with \bms
	\item ... you know what formal modelling is
	\item ... you know what predicate logic is
	\item ... you know what ProB refer to
\end{itemize}
}

\subsection{BMotion Studio}

\begin{figure}[!ht]
\begin{center}
	\includegraphics[width=14cm]{img/tutorial/bms_architecture}
	\caption{\bms~Architecture}
	\label{fig_tut_00_architecture}
\end{center}
\end{figure}

Figure~\ref{fig_tut_00_architecture} shows an overview of the architecture of \bms. 
In \bms, a visualisation is described by a visualisation template that contains visual elements and observers. 
Visual elements may be, for instance, shapes or images that represent some aspects of the model. 
For example, when modelling a communication protocol, circles may be used for representing the communicating entities of the protocol and arrows for the message exchanges between the entities. 
\bms~supports web technologies like Scalable Vector Graphics (SVG)\footnote{\url{http://www.w3.org/TR/SVG11}} and Cascading Style Sheets (CSS)\footnote{\url{http://www.w3.org/TR/css-2010}} for this purpose. 
SVG is an XML-based markup language for describing two-dimensional vector graphics. 
It comes with a number of visual elements like shapes, images and paths.
CSS is a language that can be used to describe the style of SVG visual elements (e.g. the colour or the dimension). 

Observers are used to link visual elements with the model. 
An observer is notified whenever a model has changed its state, i.e. whenever an event has been executed. 
In response, the observer will query the model's state and triggers actions on the linked visual elements in respect to the new state. 

\subsection{Formal Modelling}

We are concerned with \textit{formalizing specifications}.  This allows us a more rigorous analysis (thereby improving the quality) and allows the reuse of the specification to develop an implementation.  This comes at the cost of higher up-front investments.

This differs from the traditional development process. In a formal development, we transfer some effort from the test phase (where the implementation is verified) to the specification phase (where the specification in relation to the requirements is verified).

\subsection{Predicate Logic}
\label{predicate_logic}
\index{predicate logic}

Predicate logic is a mathematical logic containing variables that can be quantified.

Event-B supports first-order logic which is, technically speaking, just one type of predicate logic.  

\subsection{Event-B}
\label{eventb}
\index{Event-B}

Event-B is a notation for formal modelling based around an abstract machine notation (\index{abstract machine notation}).

Event-B is considered an evolution of B (also known as Classical B). It is a simpler notation which is easier to learn and use. It comes with tool support in the form of the Rodin Platform.

\subsection{ProB Animator}
\label{prob_animator}

ProB is a validation toolset for the B method including an animator, a modelchecker and other useful tools to allow users to gain confidence in their specifications. One of the components of ProB is animation. The animation component allows the user to check the presence of desired functionality and to inspect the behaviour of a specification by "clicking through" the states of the specification. ProB also provides other useful tools such as a tool to visualize graphically any predicate as a tree or a tool for graphical state representation. Such tools, especially the tool for the graphical state representation can give a better understanding of the model.

\section{Installation and first steps}
\label{tutorial_02}

\subsection{Installation}

Start off by installing BMotion Studio. BMotion Studio is a fix part of the ProB Installation and is available as a plugin for the Rigorous Open Development Environment for Complex Systems (RODIN).

\subsubsection{Step 1: Install Rodin for the first time}

\tick{You can skip this step and go on with step 2 if you have already downloaded Rodin.}

The first step is to download Rodin. Rodin is available for download at the Rodin Download page (\ref{rodin_download})

Rodin is available for Windows, Mac OS, and Linux.  No matter which platform you use, the distribution is always packed in a zip-file (\ref{zip_file}).  Download the zip file for your system anywhere on your PC.

\info{It is recommended that you download the latest stable version.}

For a detailed documentation about Rodin we refer to the official Rodin hanbdook page \href{http://handbook.event-b.org}{http://handbook.event-b.org}.

\subsubsection{Step 2: Install BMotion Studio for the first time}

Step: Start Rodin and select \textsf{Open Help $\rangle$ Install New Software...} (compare with figure \ref{fig_tut_02_install1}). Select the ProB Update site (http://www.stups.uni-duesseldorf.de/prob\_updates) and install ProB for Rodin2 Click on the \textsf{$Next >$} button and follow the installation instruction.

\begin{figure}[!h]
\begin{center}
	\includegraphics{img/tutorial/tut_02_install1.png}
	\caption{Install New Software for Eclipse or Rodin respectively}
	\label{fig_tut_02_install1}
\end{center}
\end{figure}

\subsection{The First Visualization: A Waterboiler Model}

\tick{\textbf{Goals:} The objective of this section is to learn how to create a visualization for a Event-B of a waterboiler model. The waterboiler has functions to open/close its cap, to fill/effuse water and to switch on/off.}

\subsubsection{Waterboiler Model}

Since BMotion Studio allows to create a visualization for existing Event-B models, we have to import the initial Event-B model of the waterboiler: 

\begin{enumerate}
	\item Step: Download the \file{Waterboiler.zip}{Waterboiler.zip}.
	\item Step: Select \textsf{File $\rangle$ Import}. A window should popup as shown in figure \ref{fig_tut_03_waterboiler1}.
	\item Step: Select \textsf{General $\rangle$ Existing Project into Workspace} and click on the \textsf{$Next >$} button.
	\item Step: Mark \textsf{Select archive file}, select the \texttt{Waterboiler.zip} file and click on the \textsf{Finish} button. 
\end{enumerate}

\begin{figure}[!h]
\begin{center}
	\includegraphics{img/tutorial/tut_03_waterboiler1.png}
	\caption{The Eclipse import wizard}
	\label{fig_tut_03_waterboiler1}
\end{center}
\end{figure}

You should see the Waterboiler project in your workspace now.

\subsubsection{Creating a new BMotion Studio Visualization}

Open the context menu of the Lift Event B project and select \textsf{New $\rangle$ Other}. This should bring the project wizard of Eclipse where you can create other files/projects beside a new BMotion Studio Visualization. Select \textsf{BMotion Studio $\rangle$ BMotion Visualization} and click on the \textsf{$Next >$} button. as shown in figure \ref{fig_tut_03_waterboiler2}. 

\begin{figure}[!h]
\begin{center}
	\includegraphics{img/tutorial/tut_03_waterboiler2.png}
	\caption{Creating a new BMotion Studio Visualization}
	\label{fig_tut_03_waterboiler2}
\end{center}
\end{figure}

This should bring up the \textsf{New BMotion Studio Visualization wizard} where you have to enter the relevant data needed for creating a new visualization as shown in figure \ref{fig_tut_03_waterboiler3}. The first required field is the Project name field which allows to select a project. Since, we opended the New BMotion Studio Visualization wizard via an exisisting Event B project, the corresponding project should be preselected. In the next field (BMotion Studio Visualization filename) you have the option to enter a valid, i.e. non empty and non-existing name for your visualization, i.e. "VWaterboiler". In a final step you have to select the machine for which you want to create a visualization in the table at the bottom of the wizard. With a click on the Finish button the visualization will be created and displayed in the Waterboiler Event B project folder. 

\begin{figure}[!h]
\begin{center}
	\includegraphics{img/tutorial/tut_03_waterboiler3.png}
	\caption{The New BMotion Studio Visualization wizard}
	\label{fig_tut_03_waterboiler3}
\end{center}
\end{figure}

Open the VWaterboiler file by double clicking on it. This should open the BMotion Studio editor. This part of the tutorial is finished. The next step in the tutorial is about working with Controls (\ref{tutorial_04}).

\section{Working with Controls}
\label{tutorial_04}

A Control is a graphical representation of some aspects of the model. Typically it uses images, labels, shapes or buttons to represent information. For instance, if we model a traffic light with the two colors green and red, we might simply use a circle shape for each color.

Since we want to visualize a waterboiler we have to add some nice images which represent the different states of the model like the open/closed cap. Note that this step is optional since we could use the other predefined Controls like shapes to represent the model. However, we want to cover all features in this tutorial. For this reason we created the images in order to represent the different states of the cap or the waterboiler respectively as shown in table \ref{table_tut_04_waterboilerimages}.

\begin{table}[h!]
\begin{center}
    \begin{tabular}{ | c | c |}
    \hline
	\includegraphics{img/tutorial/tut_04_capclosed.jpg} & \includegraphics{img/tutorial/tut_04_capopen.jpg} \\ \hline
	Image representing the closed cap & Image representing the open cap  \\ \hline
	\includegraphics{img/tutorial/tut_04_capfull.jpg} & \includegraphics{img/tutorial/tut_04_capon.jpg} \\ \hline
	Image representing the full cap & Image representing the waterboiler after switching it on   \\ \hline
    \end{tabular}
\caption{Images representing the different states of the Waterboiler model}
\label{table_tut_04_waterboilerimages}
\end{center}
\end{table}

In order to use the images in our visualization we have to add them to our project. This is done with the help of the Library View (\ref{library_view}) which allows us to administrate resources like images with useful features like preview and import/delete functions. 

\subsection{Systems Development}

Text ...

\input{tutorial-04}
\section{Before Getting Started}
\label{tutorial_05}

Before we get started with the actual tutorial, we are going to go over the required background to make sure that you have a rudimentary understanding of the necessary concepts.

\tick{\textbf{You can skip this section, if...}
\begin{itemize}
	\item ... you know what formal modelling is
	\item ... you know what predicate logic is
	\item ... you know what Event-B and Rodin refer to
	\item ... you know what Eclipse is
\end{itemize}
}

\subsection{Systems Development}

Text ...

\input{tutorial-06}


\chapter{Reference}
\label{reference}

\section{The BMotion Studio API}
\label{sec:bmsapi}

(tbd)

\section{Observers}
\label{sec:observers}

Observers are used to link visual elements with the model. 
An observer is notified whenever a model has changed its state, i.e. whenever an event has been executed. 
In response, the observer will query the model's state and triggers actions on the linked visual elements in respect to the new state. 
In general, observers are defined in the Groovy script file.
\bms~comes with some predefined observers that are described in the following sections.

\subsection{Transform Observer}
\label{sec:transform_observer}

The transform observer supports changing attributes of visual elements based on the current state of the animation.
The following code snippet demonstrates the basic use of a transform observer:

\begin{lstlisting}[float=ht,language=Groovy]
transform("#myvisualelement") {
    set "fill", "green"
    set "stroke", "red"
    register(bms)
}
\end{lstlisting}

In line 1 we define a jQuery selector to select the visual elements which should be transformed.
In this case we select a visual element with the id \textit{myvisualelement} (in jQuery the prefix ``\#'' denotes that we want to select and element by its id).
jQuery provides several possibilities to select visual elements.

\info{The jQuery selector API documentation\footnote{\url{http://api.jquery.com/category/selectors/}} provides an overview and a detailed documentation about selectors.}

Line 2 to 3 are actions that are made on the visual elements which are matched by the defined selector.
In this case the \textit{fill} attribute is set to the value \textit{green} and the \textit{stroke} attribute to \textit{red}.

In line 4 we register the observer to the current visualisation.
A registered observer is triggered after every state change, e.g. after executing an event.

\subsubsection{The Use of Groovy Closures}

As we use the Groovy Scripting language for defining the observers, we can make use of the entire function and feature range of it.
Transform observers supports the use of closures for defining the selector or the value of an action.

The following code snippet demonstrates the use of closures for transform observers:

\begin{lstlisting}[float=ht,language=Groovy]
transform("#myvisualelement") {
    set "fill", { (bms.eval("1 < x").value == "TRUE") ? "white" : "lightgray" }
    set "y", {
        switch (bms.eval("x").value) {
            case "0": "15"
                break
            case "1": "20"
                break
            case "2": "25"
                break
            default: "0"
        }
    }
    register(bms)
}
\end{lstlisting}

In Groovy a closure is encapsulated in curly brackets.
Line 2 and 3 show two examples for using closures for defining the value of the attribute \textit{fill} and \textit{y} respectively.
Closures are evaluated after every state change in consequence of triggering the transform observer.
As an example, in line 2 the closure evaluates the expression $1 < x$.
The value of the \textit{fill} attribute is set to \textit{white} whenever the expression evaluates to \textit{TRUE} and otherwise to \textit{lightgray}.


\subsection{Method Observer}
\label{sec:method_observer}

The following code snippet gives an example of a method observers:

\begin{lstlisting}[float=ht,language=Groovy]
callMethod("mymethod") {
    data([foo: "bar"])
    register(bms)
}
\end{lstlisting}

\subsection{Custom Observer}
\label{sec:custom_observers}

The following code snippet gives an example of a custom observer:

\begin{lstlisting}[float=ht,language=Groovy]
def customObserver = [
        apply: { bms ->
            System.out.println("Triggering custom observer.")
        }
] as BMotionObserver
bms.registerObserver(customObserver)
\end{lstlisting}

\chapter{Frequently Asked Questions}
\label{faq}

\section{How to Build a Standalone Visualisation}

Clone the BMotion Studio for ProB standalone Github repository\footnote{\url{https://github.com/ladenberger/bmotion-prob-standalone}} and put your visualisation into the folder ``resources/workspace''.
In the root folder run the following command, where XXX is the path to the html template file in the ``resources/workspace'' folder (e.g. ``myvis/vis.html''):
\begin{lstlisting}[language=bash]
gradle -Pvisualisation="XXX" buildAll
\end{lstlisting}

If you don't have gradle installed, you can use the gradlew script provided:
\begin{lstlisting}[language=bash]
./gradlew -Pvisualisation="XXX" buildAll
\end{lstlisting}

This should build the binaries without a gradle installation on your computer.
The gradle script will produce a zipped standalone version for all platforms. The zip files are located in the build/distributions folder.





\clearpage
\addcontentsline{toc}{chapter}{Index} 
\printindex

\end{document}

