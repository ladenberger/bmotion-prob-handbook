\ifdefined\isinprint
\documentclass[twoside,10pt]{book}
\usepackage[twoside,paperwidth=155.93mm,paperheight=233.89mm,hmargin={15mm,15mm},vmargin={20mm,20mm},bindingoffset=5mm]{geometry}
\usepackage[hyphens]{url} 
\usepackage[colorlinks=false,
            pdfborder={0 0 0}
	        pdftex,
            plainpages=false,
            pdfauthor={Lukas Ladenberger (Editor)},
            pdftitle={BMotionWeb for ProB Handbook},
            pdfsubject={BMotionWeb},
            pdfkeywords={ProB, Event-B, CSP, Visualization, Tool},
            pdfproducer={http://www.stups.hhu.de/ProB/index.php5/BMotion_Studio},
            pdfcreator={plastex-based tool chain}]{hyperref}
\else
\documentclass[12pt]{book}
\usepackage[left=2.5cm,top=3cm,right=2.5cm,bottom=3cm]{geometry}
\usepackage[hyphens]{url} 
\usepackage[colorlinks=true,
	        pdftex,
            plainpages=false,
            pdfauthor={Lukas Ladenberger (Editor)},
            pdftitle={BMotionWeb for ProB Handbook},
            pdfsubject={BMotionWeb},
            pdfkeywords={ProB, Event-B, CSP, Visualization, Tool},
            pdfproducer={http://www.stups.hhu.de/ProB/index.php5/BMotion_Studio},
            pdfcreator={plastex-based tool chain}]{hyperref}
\fi
\usepackage{graphicx}
\usepackage{bsymb}
\usepackage{b2latex}
\usepackage{fancyhdr,lastpage,color}
\usepackage{verbatim}
\usepackage{wrapfig}
\usepackage{makeidx}
\usepackage{fix-cm}
\usepackage[utf8]{inputenc}

\usepackage{listings}

\usepackage{longtable}

\lstdefinelanguage{Groovy}% 
{morekeywords={abstract,any,as,boolean,break,byte,case,catch,char,class, 
const,continue,def,default,do,double,else,extends,false,final,finally, 
float,for,goto,if,implements,import,instanceof,in,int,interface,label, 
long,native,new,null,package,private,protected,public,return,short, 
static,strictfp,super,switch,synchronized,this,throw,throws,transient, 
true,try,void,volatile,while,with},% 
sensitive=true,% 
morecomment=[l]//,% 
morecomment=[s]{/}{/},% 
morestring=[b]",% 
morestring=[b]',% 
}[keywords,comments,strings]

\lstdefinelanguage{B}% 
{morekeywords={MACHINE, REFINEMENT,OPERATIONS,INITIALISATION,END,BEGIN,PRE,IF,THEN,ELSE,VARIABLES,INVARIANT,SETS},% 
sensitive=true,% 
morecomment=[l]//,% 
morecomment=[s]{/}{/},% 
morestring=[b]",% 
morestring=[b]',% 
}[keywords,comments,strings]

\lstdefinelanguage{JavaScript}{
  basicstyle=\ttfamily\small,
  keywords={typeof, new, true, false, catch, function, return, null, catch, switch, var, if, in, while, do, else, case, break},
  keywordstyle=\color{magenta}\bfseries,
  ndkeywords={class, export, boolean, throw, implements, import, this},
  ndkeywordstyle=\color{magenta}\bfseries,
  identifierstyle=\color{black},
  sensitive=false,
  comment=[l]{//},
  morecomment=[s]{/*}{*/},
  commentstyle=\color{black}\ttfamily,
  stringstyle=\color{blue}\ttfamily,
  morestring=[b]',
  morestring=[b]"
}

\lstset{
  columns=fullflexible,
  numberbychapter=false,
  frame=top,frame=bottom,
  basicstyle=\ttfamily\small,    % the size of the fonts that are used for the code
  stepnumber=1,                           % the step between two line-numbers. If it is 1 each line will be numbered
  numberstyle=\tiny,
  numbersep=5pt,                         % how far the line-numbers are from the code
  tabsize=2,                              % tab size in blank spaces
  extendedchars=true,                     %
  breaklines=true,                        % sets automatic line breaking
  captionpos=b,                           % sets the caption-position to top
  mathescape=false,
  stringstyle=\color{blue}\ttfamily, % Farbe der String
  keywordstyle=\color{magenta}\ttfamily, % Farbe der String
  showspaces=false,           % Leerzeichen anzeigen ?
  showtabs=false,             % Tabs anzeigen ?
  xleftmargin=17pt,
  framexleftmargin=17pt,
  framexrightmargin=17pt,
  framexbottommargin=5pt,
  framextopmargin=5pt,
  showstringspaces=false,      % Leerzeichen in Strings anzeigen ?
  escapechar=\%,
  numbers=left,                    % where to put the line-numbers; possible values are (none, left, right)
  numbersep=5pt
 }

% Rodin Handbook Version Path. "current" is the newest version of the handbook. This should be changed if we want to build a handbook for another (i.e. older version)
\newcommand{\versionpath}{nightly}

% Absolute path to handbook
\newcommand{\handbookpath}{http://nightly.cobra.cs.uni-duesseldorf.de/bmotion/bmotion-prob-handbook}

\newcommand{\bms}{BMotion Studio}

% We generate an index
\makeindex


% defining a if(plastex) environment
\newif\ifplastex
\plastexfalse

\def\doculist#1#2{
\begin{quote}
\hspace{-10mm}
\textrm{\includegraphics[width=7mm]{#2}} % Hack!  We "mark" the image with textrm so that we can use a different CSS-Style in plastex.
\vspace{-8mm}

#1
\end{quote}
}

\def\tick#1{\doculist{#1}{img/tick_64.png}}
\def\info#1{\doculist{#1}{img/info_64.png}}
\def\warning#1{\doculist{#1}{img/warning_64.png}}
\def\pencil#1{\doculist{#1}{img/pencil_64.png}}

% macro for icons
\def\icon#1{
\includegraphics[]{img/icons/#1}
}

% macro for image versions (pdf version + html version)
% #1 Path to image for pdf version
% #2 Path to image for html version
% #3 Caption
% #4 Label
\def\imagedpi#1#2#3#4#5{
	\ifplastex
		\begin{figure}[!h]
		\begin{center}
			\includegraphics{#3}
			\caption{#4}
			\label{#5}
		\end{center}
		\end{figure}
	\else
		\begin{figure}[!h]
		\begin{center}
			\includegraphics[width=#2]{#1}
			\caption{#4}
			\label{#5}
		\end{center}
		\end{figure}
	\fi
}

% different method to write an ASCII backslash for plastex and normal pdflatex
\ifplastex
  \newcommand{\mybackslash}{\textbackslash}
\else
  % we do not use textbackslash for latex, because it does not use the current font setting
  \newcommand{\mybackslash}{\symbol{`\\}}
\fi

% Path to resources like zip's with machines
% We use a relative path in the html + eclipe version (in order to work offline)
% and an absolute path in the pdf version
\ifplastex
	\newcommand{\filepath}{files/}
\else
	\newcommand{\filepath}{\handbookpath/\versionpath/files/}
\fi

% Use this definition to create a link to the file. The definition takes to arguments. 
% The first argument (1) defines the file name i.e. Celebrity.zip or in case if you saved 
% the file in a subdirectory subdirecotry/Celebrity.zip. The second argument (2) defines 
% the name which should be displayed in the document, i.e. Celebrity Problem Example Download
\def\file#1#2{
\href{\filepath#1}{#2}
}

% We want to mark contributions from other plugins in a special way, by including the plugin's
% icon and by putting the content in a gray box.  We have to approach this differently for
% Latex and for Plastex:
% Latex: We use "shaded" from package "framed"
% Platexte: We use "verse" as the marker and create the shading with the style sheet.
\newcommand{\tmpName}{Dummy}
\ifplastex
\newenvironment{rodin-plugin}[2]
{
\renewcommand{\tmpName}{#2}
  \begin{verse}
\begin{wrapfigure}{l}{}
    \includegraphics{#1}
\end{wrapfigure}
}
{
\newline
\textit{This contribution requires the \textbf{\tmpName} plugin.  The content is maintained by the plugin contributors and may be out of date.}
\end{verse}
}
\else
\usepackage{framed}
\definecolor{shadecolor}{rgb}{0.93,0.93,0.93}
\newenvironment{rodin-plugin}[2]
{
\renewcommand{\tmpName}{#2} % Otherwise we cannot use #2 in the end block - stupid!
\begin{shaded}
\begin{wrapfigure}{l}{10mm}
\vspace{-5mm}
\includegraphics[width=10mm]{#1}
\vspace{-5mm}
\end{wrapfigure}
\noindent
}
{
\vspace{1mm}
\noindent\rule{\textwidth}{.1pt}
\vspace{1mm}
\noindent
{\scriptsize This contribution requires the \textbf{\tmpName} plugin.  The content is maintained by the plugin contributors and may be out of date.}

\end{shaded}
}
\fi

% Marginpars are  cropped - this formats them nicely.
\let\oldmarginpar\marginpar
\renewcommand\marginpar[1]{\-\oldmarginpar[\raggedleft\scriptsize{#1}]
{\raggedright\small{#1}}}
\marginparwidth=2cm

% A command to typeset names of an Event-B section (like variables, invariant, etc)
% consistently.
\newcommand{\eventbsection}[1]{\textsl{#1}}

% A command to typeset consistently the names of proof obligations
\newcommand{\eventbpo}[1]{\textsf{#1}}

% Commands for the structure of the reference section

% rrnames is used for the array of operator symbols and description
% at the beginning of a reference section.
% The environment defines an array with three columns:
% 1) The mathematical symbol
% 2) The ASCII representation
% 3) A description of the operator
\newenvironment{rrnames}%
  {\begin{tabular}{l@{\quad---\quad}l@{\quad---\quad}l}}%
    {\end{tabular}}

% The environment rodinrefentry is used for a reference section with several
% entries: Description, Definition, Types, Well-Definedness
% \rrindent is the indention in such an environment
\newlength{\rrindent}
\setlength{\rrindent}{8em}
\newenvironment{rodinrefentry}{%
   \renewcommand\descriptionlabel[1]{\makebox[\rrindent][r]{\textbf{##1}}}
   \setlength{\leftmargini}{\rrindent}
   \begin{description}%
}{%
   \end{description}%
}
\newcommand{\rrdesc}{\item[Description]}
\newcommand{\rrdef}{\item[Definition]}
\newcommand{\rrtypes}{\item[Types]}
%\newcommand{\rrwd}{\item[Well-Definedness]}
\newcommand{\rrwd}{\item[WD]}
\newcommand{\rrfis}{\item[Feasibility]}

\newcommand{\actfis}{\mathcal{F}}

% operators (L and D) for well-definendness
\newcommand{\wdl}{\mathcal{L}}
\newcommand{\wdd}{\mathcal{D}}

% a placeholder symbol for operators
\newcommand{\opelipse}{\mathbin{\Box}}

\newcommand{\podef}[4]{%
  \begin{center}
    \setlength{\parindent}{2em}\vspace{0.2em}
    \begin{tabular}{rp{0.5\textwidth}}
      \hline
      & \textbf{#1} \\
      Name       & #2 \\
      Goal       & #3 \\
      Hypotheses & #4 \\
      \hline
    \end{tabular}
  \end{center}
}

% a second approach to proof obligations
\newcommand{\pode}[3]{%
  \begin{center}
    \setlength{\parindent}{2em}\vspace{0.2em}
    \begin{tabular}{rp{0.6\textwidth}}
      \hline
      & \textbf{#1} \\
      Name       & #2 \\
      Goal       & #3 \\
      \hline
    \end{tabular}
  \end{center}
}

%%% Local Variables: 
%%% mode: latex
%%% TeX-master: "rodin-doc"
%%% End: 


\title{BMotionWeb for ProB Handbook}
\author{
Work in Progress\\
%Handbook $ $Rev: 16185 $ $ \\
\\
\href{mailto:ladenberger@cs.uni-duesseldorf.de}{ladenberger@cs.uni-duesseldorf.de}\\
\href{http://www.stups.hhu.de/ProB/index.php5/BMotion_Studio}{http://www.stups.hhu.de/ProB/index.php5/BMotion_Studio}
}

\begin{document}        
\pagestyle{empty}
\ifplastex
\maketitle
\else
\pagenumbering{roman}
\begin{titlepage}
\AddToShipoutPicture*{\BackgroundPic}
\vspace*{\titletop}
{\titledimrodin\selectfont \bfseries BMotionWeb}

\vspace*{\titlesubtitledistance}
{\titledimhandbook\selectfont \bfseries Handbook}

\vspace*{\titlesecdistance}
{\titledimsubtext\selectfont \textbf{\textsf{ProB Edition}}}

\vspace*{\titlesecdistance}
{\titledimsubtext\selectfont \textsf{Lukas Ladenberger (Editor)}}

\vspace*{\titlesecdistance}
{\titledimsponsor\selectfont %
  $\hbox{\includegraphics[height=4ex]{img/advance-logo.png}}$
  \textsf{This work is sponsored by the ADVANCE Project}}

\vspace*{\titlebottom}

\end{titlepage}
\ifdefined\isinprint
\cleardoublepage
% Title page
\vspace*{10em}
\begin{center}
  {\huge \textbf{BMotionWeb User's Handbook}}\\
  \vspace{2em}
  {\large ProB Edition}
\end{center}
\clearpage
% Copyright page
\vspace*{\fill}
\noindent\textbf{BMotionWeb User's Handbook}\\
~\\
This work, except the cover image, is licensed under the Creative Commons Attribution-ShareAlike 3.0 Unported License. To view a copy of this license, visit \href{http://creativecommons.org/licenses/by-sa/3.0/}{creative\-com\-mons.org/\-licenses/\-by-sa/3.0/} or send a letter to Creative Commons, 444 Castro Street, Suite 900, Mountain View, California, 94041, USA.\\
The cover image of a Rodin statue was created by Miikka Skaffari.
It is licensed under a Creative Commons Attribution-NonCommercial 3.0 Unported License.  To view a copy of this license, visit \href{http://creativecommons.org/licenses/by-sa/3.0/}{creative\-com\-mons.org/\-licenses/\-by-sa/3.0/} or send a letter to Creative Commons, 444 Castro Street, Suite 900, Mountain View, California, 94041, USA.
\vspace{10em}
\else\fi
\cleardoublepage
\pagenumbering{arabic}
\phantomsection
\addcontentsline{toc}{chapter}{Contents}
\tableofcontents
\fi

\chapter{Introduction}
\label{introduction}

\section{Overview}

This handbook consists of five parts:

\begin{description}
	\item[Introduction (Chapter~\ref{introduction})] You are reading the introduction right now.  Its purpose is to help you orient yourself and to find information quickly.
	\item[First Steps (Chapter~\ref{first_steps})] If you are completely new to BMotionWeb, this section is a good way to get up to speed quickly. 
	It guides you through the installation and usage of the tool.
	\item[BMotionWeb for Event-B and Classical-B (Chapter~\ref{bms4b})] This section provides comprehensive documentation of BMotionWeb for creating visualisations of Event-B or Classical-B models.
	\item[BMotionWeb for CSP (Chapter~\ref{bms4csp})] This section provides comprehensive documentation of BMotionWeb for creating visualisations of CSP-M models.	
	\item[Frequently Asked Questions  (Chapter~\ref{faq})] Common issues are listed by category in the FAQ.
	\item[Index] We included an index particularly for the print version of the handbook, but it may be useful in the electronic versions as well.  
\end{description}

\subsection{Formats of this Handbook}
\label{handbook_formats}

The handbook comes in various formats:

\begin{description}
	\item[Online] You can access the handbook \href{http://nightly.cobra.cs.uni-duesseldorf.de/bmotion/bmotion-prob-handbook/nightly/html/index.html}{online}.
	\item[PDF] Both online versions also include a link to the \href{http://nightly.cobra.cs.uni-duesseldorf.de/bmotion/bmotion-prob-handbook/nightly/pdf/bms-doc.pdf}{PDF} version of the handbook.
\end{description}

\section{Conventions}
\label{conventions}

We use the following conventions in this manual:

\tick{Checklists and milestones are designated with a tick. Here we summarize what we want to learn or should have learned so far.}
\info{Useful information and tricks are designated by the information sign.}
\warning{Potential problems and warnings are designated by a warning sign.}
\pencil{Examples and Code are designated by a pencil.}

We use \texttt{typewriter} font for file names and directories.

We use \textsf{sans serif font} for GUI elements like menus and buttons.  Menu actions are depicted by a chain of elements, separated by ``$\rangle$'', e.g. \textsf{File $\rangle$ Open Visualization}.

\section{ADVANCE}
\label{advance}

This work has been sponsored by the Advance project\footnote{\url{http://www.advance-ict.eu/}}.  ADVANCE is an FP7 Information and Communication Technologies Project funded by the European Commission. The overall objective of ADVANCE is the development of a unified tool-based framework for automated formal verification and simulation-based validation of cyber-physical systems.

The ADVANCE project is unique in addressing both simulation and formal verification within a single design framework.

Unification is being achieved through the use of a common formal modelling language supported by methods and tools for simulation and formal verification. An integrated tool environment is providing support for construction, verification and simulation of models.

ADVANCE is building on an existing formal modelling language - Event-B - and its associated tools environment - Rodin - with strong support for formal verification. In ADVANCE, Rodin is being further strengthened and augmented with novel approaches to multi-simulation and testing.

\section{Creative Commons Legal Code}
\label{sec:cc}        

The work presented here is the result of an collaborative effort
that took many years.  To ensure that access to this work stays free
and to avoid any legal ambiguities, we decided to formally lincense
it under the Creative Commons Share-Alike License.

This work is licensed under the Creative Commons Attribution-ShareAlike 3.0 Unported License. To view a copy of this license, visit \url{http://creativecommons.org/licenses/by-sa/3.0/} or send a letter to Creative Commons, 444 Castro Street, Suite 900, Mountain View, California, 94041, USA.



% \section{Style Guide}
\label{style-guide}

\info{For now, we will manage the style guide as a \LaTeX~document together with the rest of the documentation.  We may take it out upon publication.}

\subsubsection{General Stylistic Guidelines}

\begin{itemize}
	\item The conventions (\ref{conventions}) are part of the style guide.

	\item Use the ``we'' form.

	\item We use British English.

	\item When referring to the different views in Rodin, the word ``view'' should be written in lowercase, e.g., the Rodin Problems view.
\end{itemize}

\subsubsection{Files}

Files should be saved in the \texttt{files} subdirectory. You can also create a subdirectory. Then, use the definition 

\begin{verbatim} \file{1}{2} 
\end{verbatim} 

to create a link to the file. The definition extends to arguments. The first argument (1) defines the file name i.e. \texttt{Celebrity.zip} or if you saved the file in a subdirectory \texttt{subdirectory/Celebrity.zip}. The second argument (2) defines the name which should be displayed in the document, i.e. ``Celebrity Problem Example Download''.

\warning{Please note, that you only enter the file name without a path before (expected subdirectories). The build script assigns the correct path to the file on the server automatically .}

Here is an example using the definition: \file{Celebrity.zip}{Celebrity Problem Example Download}.

\subsubsection{Avoiding Redundancy}

We will reduce (or avoid) redundancy through heavy linking, following these guidelines:

\begin{itemize}
	\item If in doubt, provide the bulk of the information in the Reference section.  For instance, the FAQ entry ``What is Event-B?''  Should simply refer to the Event-B entry in the Reference section.
	\item Web Links should not appear multiple times
	\item List web links as footnotes in the Tutorial and FAQ.
	\item List web links in a ``See also'' Section in the Reference.
\end{itemize}

\subsubsection{Sections}

\begin{itemize}
	\item If referring to a specific chapter or section, use uppercase to denote it, e.g. ``in Chapter~3''.
	\item We also refer to subsections as Section, e.g. ``see Section~2.5''
	\item We have a small number of well-defined chapters which are the top level structuring element.
	\item Sections and subsections are numbered.  In the HTML-Versions, they are broken into subpages.
    \item Subsubsections do not receive numbers and are not broken into subpages in the HTML.  Keep this in mind regarding both the reading flow and page sizes.
	\item Avoid linking (ref) to subsubsections, as they don't have a number.  Latex will instead provide a link to the next higher element.  This works, but it could create confusion.
	\item Generally, we should avoid gaps in the hierarchy (i.e. having a subsubsection in a section without a subsection in between).\footnote{Coincidentally, this style guide violates this rule. Reason: We want the style guide to remain whole and not be broken into subsections, but the proper hierarchy is a section.}
	\item Section labels should be all in lower case. Use ``\_'' for blanks.
	\item We use the prefix ``\texttt{int\_}'' for introduction section labels, ``\texttt{tut\_}'' for tutorial section labels following a meaningful description of the section (i.e. \texttt{tut\_rodin\_installation}) and ``\texttt{faq\_}'' for faq section labels following a short version of the title (i.e. \texttt{faq\_diff\_eventb\_b}). Reference section labels have no prefix.
\end{itemize}

\subsubsection{Images}
\begin{itemize}
	\item Images must be no more than 700 pixels in width (for HTML version) and no more than 160 mm (for PDF version). This is fairly easy for bitmaps (screenshot), pay attention to this regarding how plasTeX converts vector images. (see Latex section below on how to include images)

	\item Screenshots should look neat and consistent.  Horizontal real estate will always be an issue, so please resize the windows before taking the screenshot to keep things readable at 700 pixel width.  (see Latex section below on how to include images)

	\item Images should always have a caption and a label. Please use the same conventions for the figure labels that are used for the the section labels, but use the prefix ``\texttt{fig\_}''. \\Example: ``\texttt{fig\_tut\_03\_traffic\_light}''. For referencing do not use references like "As shown bellow" or "Like in the following image:". Use always a format like:

\begin{verbatim}"... as shown in figure \ref{fig_tut_03_traffic_light}."\end{verbatim}

	\item We use icons from Pixel-Mixer, which are free as long as credit is given: \url{http://www.softicons.com/free-icons/toolbar-icons/basic-icons-by-pixelmixer}

  \item We include Window decoration only when it is really necessary.  If we discuss only some views, we crop the rest away.  Please crop neatly, following edges. Even if you need a screenshot with window decoration, you should always use the same Window decoration (i.e. linux ubuntu default decoration style). If you need such a screenshot, please contact Lukas.

  \item Image file names should be all in lower case and not include umlaute or special characters. Use ``\_'' for blanks.

  \item Image files should keep the following rules:
	
\begin{itemize}
		\item Tutorial images should be saved in the sub folder \texttt{img/tutorial} with the prefix ``\texttt{tut\_}'' following the section number. For instance, \texttt{tut\_01\_image1.png}.

		\item FAQ images should be saved in the sub folder \texttt{img/faq} with the prefix ``\texttt{faq\_}''. For instance, \texttt{faq\_image1.png}.

	\end{itemize} 
	\end{itemize}

\subsubsection{Icons}

For icons (i.e. RODIN platform icons) in continuous text use the command: 

\begin{verbatim} \icon{1} \end{verbatim} 

The first (1) argument takes the path to the icon (i.e. \texttt{rodin/auto\_prover.png}). You can find all icons in the folder \texttt{latex/img/icons/}

Here is an example using icons in continuous text:

\textit{Lorem ipsum dolor sit amet, consetetur \icon{rodin/auto_prover.png} sadipscing elitr, sed diam nonumy eirmod tempor invidunt ut labore et dolore magna aliquyam erat, sed diam voluptua. At vero eos et accusam et justo duo dolores et ea rebum. Stet clita kasd gubergren, no sea takimata sanctus est Lorem ipsum dolor sit amet. Lorem ipsum dolor sit amet, consetetur sadipscing elitr, sed diam nonumy eirmod tempor invidunt ut labore et dolore \icon{rodin/lasoo_prover.png} magna aliquyam erat, sed diam voluptua. At vero eos et accusam et justo duo dolores et ea rebum. Stet clita kasd gubergren, no sea takimata sanctus est Lorem ipsum dolor sit amet.}

\subsubsection{Index}
Please add meaningful entries to the index. This helps users to look up information more efficiently.
The \LaTeX-command to do this is \verb#\index{#\textit{word}\verb#}#.
\begin{itemize}
\item The most important thing to think about when considering the index is the user perspective: 
  What is the word that a user will look up to try and find this topic?
\item An index entry should be a noun, singular and written in lowercase (uppercase if it's a name).
\item If the word appears in are several locations in the documentation (e.g. in the tutorial and the
  reference section), try to highlight the most commonly used location with \verb#\index{word|textbf}#.
  This is usually an entry in the reference section.
\item Adjectives should be used with care. 
  Please use them only if omitting the adjective changes the meaning of the entry drastically or
  if the adjective is necessary to distinguish the entry from other entries with the same noun but
  other location.
\item If you use an adjective, consider using the noun as a key for sorting: \verb#\index{cloud,grey@grey cloud}#
  instead of \verb#\index{grey cloud}#
\item If you have multiple entries for the same word, consider using sub-entries (\verb#\index{cloud!grey}#)
  if different things are meant. Another example is \verb#\index{true!as predicate}# and \\ \verb#\index{true!as expression}#
\item Multiple index entries for the same text block are alright.
\item If you use the index command, please have a look at the result.
\end{itemize}

\subsubsection{\LaTeX{} Styling}

\begin{itemize}
	\item Try to avoid fancy \LaTeX formatting, as PlasTeX (used for generating HTML) is temperamental.  Macros are especially problematic, and sometimes the result is just ugly.
	\item We have the option to use different files to generate the PDF and HTML, but we would generally prefer not to do this.  Look at \texttt{bsymb.sty} and \texttt{plastex-bsymb.sty} as an example. \textbf{NOTE:} We don't do this any more for any style files.
	\item Every section should have a label, reflecting the section name, all lowercase, spaces replaced with underscores (\_).
	\item Don't create subdirectories in the \texttt{latex} folder, as the scripts cannot always deal with them.
	\item Put images in the \texttt{img} folder.  Feel free to create additional directory structures underneath.
	\item Files other than images (e.g. Event-B projects) --- we have macros to link them absolute in the PDF, and to link them relative in HTML and Eclipse Help.
	\item Use linked sections numbers (generated with \texttt{\\ref{}} instead of hyperlinks for cross-references. This is necessary for the print documentation to be useful.
	\item When including images in Latex, do not provide a width!  Instead, try to embed the print size in the image itself.  For instance, PNGs allow you to set the print size (in mm).  This way we can be sure that the images are rendered as HTML without distortion.
\end{itemize}

\subsubsection{Contributions from Plugin Developers}
\label{sec:plugin_contributions}

\begin{rodin-plugin}{prob.png}{ProB}
We want to encourage Plugin developers to contribute to the Handbook, but we have to make it clear that we cannot maintain that documentation.  Therefore, it has to be clearly marked.  We use a custom environment for that purpose that
(1) provides the Plugin's icon, (2) adds a disclaimer to the end of the custom documentation and (3) puts the content into a gray box (like this one).

There are some limitations to the environment that we use.  Specifically, headlines and images should be kept outside the gray box.

\end{rodin-plugin}

%%% Local Variables: 
%%% mode: latex
%%% TeX-master: "rodin-doc"
%%% End: 


\section{Background}

\subsection{BMotionWeb}

\begin{figure}[!ht]
\begin{center}
	\includegraphics[width=14cm]{img/tutorial/bms_architecture}
	\caption{\bms~Architecture}
	\label{fig_tut_00_architecture}
\end{center}
\end{figure}

Figure~\ref{fig_tut_00_architecture} shows an overview of the architecture of \bms. 
In \bms, a visualisation is described by a visualisation template that contains visual elements and observers. 
Visual elements may be, for instance, shapes or images that represent some aspects of the model. 
For example, when modelling a communication protocol, circles may be used for representing the communicating entities of the protocol and arrows for the message exchanges between the entities. 
\bms~supports web technologies like Scalable Vector Graphics (SVG)\footnote{\url{http://www.w3.org/TR/SVG11}} and Cascading Style Sheets (CSS)\footnote{\url{http://www.w3.org/TR/css-2010}} for this purpose. 
SVG is an XML-based markup language for describing two-dimensional vector graphics. 
It comes with a number of visual elements like shapes, images and paths.
CSS is a language that can be used to describe the style of SVG visual elements (e.g. the colour or the dimension). 

Observers are used to link visual elements with the model. 
An observer is notified whenever a model has changed its state, i.e. whenever an event has been executed. 
In response, the observer will query the model's state and triggers actions on the linked visual elements in respect to the new state. 

\subsection{Formal Modelling}

We are concerned with \textit{formalizing specifications}.  This allows us a more rigorous analysis (thereby improving the quality) and allows the reuse of the specification to develop an implementation.  This comes at the cost of higher up-front investments.

This differs from the traditional development process. In a formal development, we transfer some effort from the test phase (where the implementation is verified) to the specification phase (where the specification in relation to the requirements is verified).

\subsection{Predicate Logic}
\label{predicate_logic}
\index{predicate logic}

Predicate logic is a mathematical logic containing variables that can be quantified.

Event-B supports first-order logic which is, technically speaking, just one type of predicate logic.  

\subsection{Event-B}
\label{eventb}
\index{Event-B}

Event-B is a notation for formal modelling based around an abstract machine notation (\index{abstract machine notation}).

Event-B is considered an evolution of B (also known as Classical B). It is a simpler notation which is easier to learn and use. It comes with tool support in the form of the Rodin Platform.

\subsection{Classical-B}
\label{classicalb}
\index{Classical-B}

\subsection{CSP}
\label{csp}
\index{CSP}

CSP is a notation used mainly for describing concurrent and distributed systems.
There are two major CSP dialects: CSP-M and CSP\#.
In BMotionWeb, we concentrate on the creation of domain specific visualisations for CSP-M models.

\subsection{ProB Animator}
\label{prob_animator}

ProB is a validation toolset for the B method including an animator, a modelchecker and other useful tools to allow users to gain confidence in their specifications. One of the components of ProB is animation. The animation component allows the user to check the presence of desired functionality and to inspect the behaviour of a specification by "clicking through" the states of the specification. ProB also provides other useful tools such as a tool to visualize graphically any predicate as a tree or a tool for graphical state representation. Such tools, especially the tool for the graphical state representation can give a better understanding of the model.

\chapter{First Steps}
\label{first_steps}
\section{Installation and Start}
\label{installation}

Start off by downloading \bms~for your operating system. 
You can find the latest version of the tool at \url{http://www.stups.hhu.de/ProB/index.php5/BMotion_Studio}.
Decompress the archive and expand the directory if necessary. 
%\warning{Do not change the location and structure of the files and directories within the folder!}
%Navigate to the \texttt{server/bin} folder and start the server by entering the bash command:
%
%\begin{lstlisting}[language=bash]
%.\standalone
%\end{lstlisting}
%
%\info{Windows users should execute the \texttt{standalone.bat} file.}
Navigate to the application folder and start \bms~by executing the \texttt{bmotion-prob} binary.
%Now, navigate to the \texttt{client} folder and start the client by executing the \texttt{bmotion-prob} program.
After a short loading time you should see the window shown in Figure~\ref{fig_bms_client}.

\begin{figure}[!ht]
\begin{center}
	\includegraphics[width=.8\textwidth]{img/tutorial/clientstartscreen.png}
	\caption{\bms\ Client}
	\label{fig_bms_client}
\end{center}
\end{figure} 

%Your default browser should open and show the default workspace.
%The workspace contains the following predefined folders:
%\begin{itemize}
%\item \texttt{libs}: This folder contains JavaScript libraries that are needed for BMotion Studio.
%\item \texttt{b\_template}: A visualization template for creating visualizations for Classical-B and Event-B models.
%\item \texttt{csp\_template}: A visualization template for creating visualizations for CSP models.
%\end{itemize}

\section{Open a Visualization Template}
\label{open_vis_template}

To open a visualization, click on the box in the middle of the window and select the \texttt{bmotion.json} file of the visualization or just drag and drop the \texttt{bmotion.json} file into the box.
You can also open a visualization via the top menu: \textsf{File $\rangle$ Open Visualization}.

\section{Create a new Visualization Template}
\label{vis_template}

%Let's start by creating a new visualization template.
You can download the \file{bms-b-template.zip}{predefined template} as a starting point to create a new visualization template.
%The easiest way to create a new visualization template is to download the \file{bms-b-template.zip}{predefined template}.
Decompress the archive, expand the directory if necessary and navigate to the decompressed folder.
The folder contains some files:

%	duplicate one of the default templates  \texttt{b\_template} (for Event-B or Classical-B visualizations) or \texttt{csp\_template} (for CSP-M visualizations) that are included in the \texttt{workspace} folder of your \bms~installation.
%Just duplicate the folder.
%After refreshing your browser, the newly created folder should appear in your workspace.
%Navigate to the folder. 
%The folder contains three files.

%\paragraph{\texttt{template.groovy:}}
%The Groovy script file is the place where the user can communicate with the formal model by means of the ProB Java API\footnote{\url{http://www.stups.hhu.de/ProB/index.php5/ProB_Java_API}}.
%For instance, the user may register methods that can be called in the JavaScript w.
%The Groovy script file is the place where you can setup the communication between your visualization and the ProB animator.
%In particular, the Groovy script file is the link between the formal model and the visualization.
%It allows you to programmatically control the ProB animator and to access the actual formal model being visualised.

\paragraph{\texttt{bmotion.json:}}
The \texttt{bmotion.json} file is the root file of every \bms\ visualization.
It contains the configuration formatted using JSON (JavaScript Object Notation)\footnote{\url{http://www.json.org}.}.

\info{Section~\ref{sec:manifest} contains a full list of available options.}

A minimal configuration of the \texttt{bmotion.json} file should contain the path to the formal model you want to visualize.
Here is an example \texttt{bmotion.json} file with a minimal configuration:

\begin{lstlisting}[language=JavaScript]
{
  "model": "model/model.mch"
}
\end{lstlisting}


\paragraph{\texttt{script.js:}}
In the JavaScript file you can setup observers and actions (see Section~\ref{reference_b}).
Moreover, the user can take advantage of the entire JavaScript language.
There exist are a lot of libraries for JavaScript that you can apply to create custom visualizations.
For instance, it exists libraries for generating chart and plot diagrams.
%In addition, you can call functions that are registered in the Groovy script file.
%This enables you to add some interactivity to your visualization.
%For instance, pressing a button in your visualization could cause the execution of an Event-B event.
Here is an example \texttt{script.js} file with a minimal configuration:

\begin{lstlisting}[language=JavaScript]
requirejs(['bmotion.template'], function (bms) {
  // Put your code here
});
\end{lstlisting}

%The \textit{prob} parameter is the access point to the BMotion Studio for ProB API.

\paragraph{\texttt{template.html:}}
%The HTML file is the root file of your visualization. It contains the actual visualization.
The HTML file contains the actual visualization.
Here is an example \texttt{template.html} file with a minimal configuration:

\begin{lstlisting}[language=html]
<html data-bms-visualisation>
  <head>
    <title>BMotion Studio Visualization</title>
  </head>
  <body>
    <script data-main="script" src="require.js"></script>
  </body>
</html>
\end{lstlisting}

Please note the attribute \textit{data-bms-visualisation} in line 1.
Every \bms\ visualization template file should contain the empty attribute \textit{data-bms-visualisation}.
In addition, every \bms\ visualization template file should contain a reference to your \texttt{scripts.js} (see line 6).

%The meta tag \textit{bms.tool} (line 4) defines the formalism or the simulator respectively that should be used. 
%Two values are allowed: ``BAnimation'' for creating visualizations of Event-B or Classical-B models and ``CSPAnimation'' for creating visualizations of CSP-M models.
%The meta tag \textit{bms.script} (line 5) contains the link to the Groovy script file.
%Finally, in line 9 we define the path to the JavaScript file.

The user is not restricted to an editor in order to create a visualization.
The user can make use of any tool that supports the creation of SVG graphics or HTML documents.
For the tutorials (Section~\ref{tutorial_b} and~\ref{tutorial_csp}) we are going to use the Inkspace\footnote{\url{https://inkscape.org}} editor. Inkscape is an editor for creating vector graphics that is available for Windows, Mac OS X and Linux.
It's free and open source.
With Inkscape the user can export the vector graphic into the SVG format.
The SVG code can then be included in the \texttt{template.html} file.

\begin{figure}[!ht]
\begin{center}
	\includegraphics[width=12cm]{img/tutorial/tut_02.png}
	\caption{Creating an SVG graphic with Inkscape}
	\label{fig_tut_02_inkscape}
\end{center}
\end{figure} 

\paragraph{\texttt{bmotion.vis.js and require.js:}}
JavaScript libraries that are needed for running the visualization.
Please do not edit them!

%\section{Link a Model with a Visualization}
%
%In order to link a model with the visualization, open the \texttt{bmotion.json} file with an editor of your choice and adapt the \textit{model} property.
%The model \textit{model} property should contain the path to your model file (e.g. \texttt{mymodel/model.mch}).
%The model should be places relative to your bmotion.json file.
%Linking a model within the \texttt{bmotion.json} file will automatically load the model, when starting the visualization.

%To create a link between graphical elements and the model, please checkout the Section \ref{tutorial_b} for Event-B and Classical-B or \ref{tutorial_csp} for CSP.


\chapter{BMotionWeb}
\label{bms}
\chapter{Reference}
\label{reference}

\section{The BMotion Studio API}
\label{sec:bmsapi}

(tbd)

\section{Observers}
\label{sec:observers}

Observers are used to link visual elements with the model. 
An observer is notified whenever a model has changed its state, i.e. whenever an event has been executed. 
In response, the observer will query the model's state and triggers actions on the linked visual elements in respect to the new state. 
In general, observers are defined in the Groovy script file.
\bms~comes with some predefined observers that are described in the following sections.

\subsection{Transform Observer}
\label{sec:transform_observer}

The transform observer supports changing attributes of visual elements based on the current state of the animation.
The following code snippet demonstrates the basic use of a transform observer:

\begin{lstlisting}[float=ht,language=Groovy]
transform("#myvisualelement") {
    set "fill", "green"
    set "stroke", "red"
    register(bms)
}
\end{lstlisting}

In line 1 we define a jQuery selector to select the visual elements which should be transformed.
In this case we select a visual element with the id \textit{myvisualelement} (in jQuery the prefix ``\#'' denotes that we want to select and element by its id).
jQuery provides several possibilities to select visual elements.

\info{The jQuery selector API documentation\footnote{\url{http://api.jquery.com/category/selectors/}} provides an overview and a detailed documentation about selectors.}

Line 2 to 3 are actions that are made on the visual elements which are matched by the defined selector.
In this case the \textit{fill} attribute is set to the value \textit{green} and the \textit{stroke} attribute to \textit{red}.

In line 4 we register the observer to the current visualisation.
A registered observer is triggered after every state change, e.g. after executing an event.

\subsubsection{The Use of Groovy Closures}

As we use the Groovy Scripting language for defining the observers, we can make use of the entire function and feature range of it.
Transform observers supports the use of closures for defining the selector or the value of an action.

The following code snippet demonstrates the use of closures for transform observers:

\begin{lstlisting}[float=ht,language=Groovy]
transform("#myvisualelement") {
    set "fill", { (bms.eval("1 < x").value == "TRUE") ? "white" : "lightgray" }
    set "y", {
        switch (bms.eval("x").value) {
            case "0": "15"
                break
            case "1": "20"
                break
            case "2": "25"
                break
            default: "0"
        }
    }
    register(bms)
}
\end{lstlisting}

In Groovy a closure is encapsulated in curly brackets.
Line 2 and 3 show two examples for using closures for defining the value of the attribute \textit{fill} and \textit{y} respectively.
Closures are evaluated after every state change in consequence of triggering the transform observer.
As an example, in line 2 the closure evaluates the expression $1 < x$.
The value of the \textit{fill} attribute is set to \textit{white} whenever the expression evaluates to \textit{TRUE} and otherwise to \textit{lightgray}.


\subsection{Method Observer}
\label{sec:method_observer}

The following code snippet gives an example of a method observers:

\begin{lstlisting}[float=ht,language=Groovy]
callMethod("mymethod") {
    data([foo: "bar"])
    register(bms)
}
\end{lstlisting}

\subsection{Custom Observer}
\label{sec:custom_observers}

The following code snippet gives an example of a custom observer:

\begin{lstlisting}[float=ht,language=Groovy]
def customObserver = [
        apply: { bms ->
            System.out.println("Triggering custom observer.")
        }
] as BMotionObserver
bms.registerObserver(customObserver)
\end{lstlisting}

\chapter{BMotionWeb for Event-B and Classical-B}
\label{bms4b}
\section{Tutorial}
\label{tutorial_b}

The objective of this chapter is to get you to a stage where you can use BMotionWeb to visualize Event-B or Classical-B models. 
We expect that you have already downloaded the BMotionWeb tool (see Section~\ref{installation}).
 
%We expect you to have a basic understanding of logic and an idea why doing formal modelling is a good idea.  
You should be able to work through the tutorial with no or little outside help.
We encourage you not to download solutions to the examples but instead to actively build them up yourself as the tutorial progresses.
If something is unclear, remember to check the Reference (\ref{reference_b}) for more information.

\subsection{Preparation}

Let's start by creating a new visualization template as described in Section~\ref{vis_template}.
%Just download the \file{bms-b-template.zip}{predefined template}, decompress the archive and rename the folder to \texttt{lift}.
%After refreshing your browser, a new folder called \texttt{lift} should appear in your workspace (see Figure~\ref{fig_tut_01_workspace}).

%\subsection{Creating a new visualization Template}
%
%Let's start by creating a new visualization template.
%The easiest way to create a new visualization template is to duplicate the default template \texttt{b\_template} that is included in the \texttt{workspace} folder of your \bms~installation.
%Just duplicate the \texttt{b\_template} folder and rename it to \texttt{lift}.
%After refreshing your browser, a new folder called \texttt{lift} should appear in your workspace (see Figure~\ref{fig_tut_01_workspace}).
%Navigate to the \texttt{lift} folder. 
%The folder contains three files:
%\begin{itemize}
%\item \texttt{template.html}: The HTML file is the root file of your visualization. It contains the actual visualization and it's configuration.
%\item \texttt{template.groovy}: The Groovy script file is the place where the user can communicate with the formal model by means of the ProB Java API\footnote{\url{http://www.stups.hhu.de/ProB/index.php5/ProB_Java_API}}.
%For instance, the user may register methods that can be called in the JavaScript file.
%%The Groovy script file is the place where you can setup the communication between your visualization and the ProB animator.
%%In particular, the Groovy script file is the link between the formal model and the visualization.
%%It allows you to programmatically control the ProB animator and to access the actual formal model being visualised.
%%In addition, you can register functions that can be called from the visualization, e.g. executing an Event-B event after pressing a button in the visualization.
%\item \texttt{template.js}: In the JavaScript file you can setup observers and actions.
%Moreover, the user can take advantage of the entire JavaScript language.
%There exist are a lot of libraries for JavaScript that you can apply to create custom visualizations.
%For instance, it exists libraries for manipulating the DOM of an HTML document, or for generating chart and plot diagrams.
%%In addition, you can call functions that are registered in the Groovy script file.
%%This enables you to add some interactivity to your visualization.
%%For instance, pressing a button in your visualization could cause the execution of an Event-B event.
%\end{itemize}
%
%\begin{figure}[!ht]
%\begin{center}
%	\includegraphics[width=12cm]{img/tutorial/tut_01.png}
%	\caption{\bms~Workspace}
%	\label{fig_tut_01_workspace}
%\end{center}
%\end{figure}

\subsection{The Formal Model}

We are going to create a visualization	for a simple lift system that allows movement of a single lift cage between three floors.
The door of the lift can be closed and opened - all in response to the pressing of floor call and cage send buttons.

You can download the Event-B model \file{EventBLift.zip}{here}.
Decompress the archive and put the files into a new folder called \texttt{model} relative to your \texttt{index.html} file.

\subsection{Link the Model with the Visualization}

The first step consists of linking the model with the visualization.
For this, open the \bms\ manifest file with an editor of your choice and set the \textit{model} path property to ``model/MLift.bcm''.
This links the visualization with the Event-B machine called ``MLift''.
Linking a model within the \bms\ manifest file will automatically load the model, when starting the visualization (see Section~\ref{sec:stat_vis}).

\subsection{Create the Actual Visualization}

Please download the prepared \file{lift.svg}{lift.svg} file and open it with Inkscape as demonstrated in Figure~\ref{fig_tut_02_inkscape}.
Feel free to modify and explore the SVG graphic.
In order to link graphical elements of the SVG graphic with the formal model later, we have to give them identifiers. 
For this, select an element in Inkscape, open the context menu and select \textsf{Object Properties}.
A popup window should be opened as demonstrated in Figure~\ref{fig_tut_02_inkscape}.
As an example, we give the graphical element that represents the door (the gray filled rectangle), the id ``door''.
In Section~\ref{sec_creation_observers} we explain how we can use this information in order to establish \textit{observers}.
If you are satisfied with your SVG graphic, save it as a plain SVG graphic with \textsf{File $\rangle$ Save As}.
Select \textsf{Plain SVG (*.svg)} as an output format and click on the \textsf{Save} button.
You can save the SVG file anywhere on your local system. 
Open the \texttt{index.html} file with an editor of your choice and change the path to the SVG file ``lift.svg'' within the ``data-bms-svg'' attribute.
%Open the SVG file with an editor of your choice, copy the SVG code and paste it within the body tag in the \texttt{index.html} file.

\subsection{Start the Visualization}
\label{sec:stat_vis}

Let's try out the visualization for the first time!
Just drag and drop the \bms\ manifest files on the marked area or open it via the file dialog.
%In your browser, navigate to the folder \texttt{lift} folder and click on the \texttt{template.html} file.
The visualization should start.
At the top menu you will find a menu item called \textsf{ProB} for opening different ProB related views.
For instance, Figure~\ref{fig_tut_03_running1} shows the running lift visualization with the ProB Events view opened.

At the moment the appearance of the visualization doesn't change whenever a state change occurred (i.e. when executing events in the ProB Events view).
This is because no observers exist yet.
In the next Section we learn how we can link graphical elements with the formal model by establishing observers.

\begin{figure}[!ht]
\begin{center}
	\includegraphics[width=12cm]{img/tutorial/tut_03.png}
	\caption{Running the Lift visualization for the First Time}
	\label{fig_tut_03_running1}
\end{center}
\end{figure}

\subsection{Create Observers}
\label{sec_creation_observers}

Observers are used to link graphical elements with the model. 
An observer is notified whenever the model has changed its state, i.e. whenever an event has been executed. 
In response, the observer will query the model's state and triggers actions on the linked graphical elements in respect to the new state.
In general, observers are written in JavaScript and should be placed in the \texttt{script.js} file. 
As an example, consider the following \textit{formula observer}:

\begin{lstlisting}[language=JavaScript, caption={Formula Observer Displaying the Current Floor (JavaScript)}]
bms.observe("formula", {
  selector: "#txt_cur_floor",
  formulas: ["cur_floor"],
  trigger: function (origin, values) {
    origin.text(values[0])
  }
});
\end{lstlisting}

\info{Checkout Section~\ref{b_observers} for more details about observers.}

We are going to explain the JavaScript code line by line.
In line 1 we register a formula observer on the graphical element with the id \textit{txt\_cur\_floor} (line 2) that is located in our \texttt{index.html} file.
%Line 1 means that we want to transform the graphical element with the id \textit{txt\_cur\_floor} that is located in our \texttt{index.html} file.
\bms~follows the jQuery selector syntax\footnote{For more information about jQuery and selectors we refer the reader to the jQuery
API documentation http://api.jquery.com/category/selectors/.} to select graphical elements.
The prefix ``\#'' denotes that we want to select an element by its id.
In line 3 we define a list of observed formulas.
In this case we observe the variable \textit{cur\_floor}.
In line 4 to 6 we define a trigger function that is called after every state change with its \textit{origin} (the origin parameter holds a reference to the graphical element that the observer is attached to) and the \textit{values} (the values parameter contains the values of the defined formulas in an array, e.g. use \textit{values[0]} to obtain the value of the first formula).
The trigger function changes the text of the graphical element (\textit{origin}) to the current value of the variable \textit{cur\_floor} (\textit{values[0]}).
%Reload the visualization by clicking on the \texttt{Reload} button and play with the visualization by executing some events.
%Line 2 affects that the attribute \textit{text} will be set to the value that is returned by the followed closure.
%In particular, the closure evaluates the expression \textit{cur\_floor} in the current state and returns the value to be set.
%In other words, the observer sets the current value of the variable \textit{cur\_floor} into the visual text element with the id \textit{txt\_cur\_floor}.
%Line 3 is responsible to register the observer.
%All registered observer will be triggered after every state change.

Let's create another observer.
Check out the following JavaScript snippet:

\begin{lstlisting}[language=JavaScript, caption={Formula Observer for the Lift Door (JavaScript)}]
bms.observe("formula", {
  selector: "#door",
  formulas: ["cur_floor", "door_open"],
  trigger: function (origin, values) {
    
    switch (values[0]) {
      case "0":
        origin.attr("y", "175");
        break;
      case "1":
        origin.attr("y", "60");
        break;
      case "-1":
        origin.attr("y", "275");
        break;
    }
    
    if(values[1] === "TRUE") {
      origin.attr("fill", "white");
    } else {
      origin.attr("fill", "lightgray");
    }
    
  }
});
\end{lstlisting}

In line 1 we register a formula observer on the graphical element that matches the selector ``\#door'' (line 2) (similar to the previous defined formula observer).
%Line 1 means that we want to transform the graphical element with the id \textit{door}, similar to the previous observer.
In line 3 we define the set of observed formulas (\textit{cur\_floor} and \textit{door\_open}).
In line 4 to 24 we define a trigger function, that makes the following action:
Line 5 to 15 will switch the \textit{y} coordinate of the door (denoting the movement of the door between floors) according to the current value of the variable \textit{cur\_floor} (\textit{values[0]}).
Lines 18 to 22 affect that the attribute \textit{fill} of the door will be set to ``white'' (denoting the door is open) whenever the formula \textit{door\_open} evaluates to \textit{TRUE} in the current state (\textit{values[1]}), otherwise to ``lightgray'' (denoting the door is closed).
%Whenever the expression \textit{door\_open} evaluates to \textit{TRUE}, the value \textit{white} (denoting the door is opened) is returned, otherwise the value \textit{lightgray} (denoting the door is closed) is returned.
%Line 3 to 17 will switch the \textit{y} coordinate of the door (denoting the movement of the door between floors) according to the evaluation result of the expression \textit{cur\_floor}.
%In line 18 we register the observer.
%You can use the entire Groovy power and feature range for defining your observers.

Add both snippets to your \texttt{script.js} file, save the file and click on the \texttt{Reload} button.
Let's see how this affects our visualization:
Setup and initialize the machine using the ProB events view.
Execute some events and see what happens.
For instance, Figure~\ref{fig_tut_04_running2} shows the lift visualization where the lift is on floor 0 and the door is open.

\begin{figure}[!ht]
\begin{center}
	\includegraphics[width=12cm]{img/tutorial/tut_04.png}
	\caption{Lift visualization with observers}
	\label{fig_tut_04_running2}
\end{center}
\end{figure}

\subsection{Add Event Handler}

In this Section we learn how we can enhance our visualization with interactive features, e.g. executing an Event-B event by clicking on a graphical element.

\info{Checkout Section~\ref{b_event_handler} for more details about event handlers.}

Let's add an interactive feature, where the user can click on a floor label to order the lift on the corresponding floor.
Add this code snippet to your \texttt{script.js} file:
\newpage
\begin{lstlisting}[language=JavaScript, caption={Example of an Execute Event Handler (JavaScript)}]
bms.executeEvent({
  selector: "text[data-floor]",
  events: [
    {
      name: "push_call_button", 
      predicate: function (origin) {
        return "b=" + origin.attr("data-floor")
      }
    }
  ]
});
\end{lstlisting}

%For instance, clicking on the floor Label ``Floor 1'' should execute the Event-B event \textit{push\_call\_button(1)}.
In line 1 we register an execute event handler for each graphical element that matches the defined selector ``text[data-floor]'' (line 2).
In particular, the selector matches the three floor labels (see Figure~\ref{fig_tut_03_running1}).
In line 3 to 10 we define the list of events that should be wired with the graphical elements.
Every event should contain the \textit{name}.
In addition, the user may enter a \textit{predicate} that defines the event's arguments.
If the user defines more than one event, a tooltip will be shown with a list of the defined events after clicking on the graphical element.
In our example we define only one event with the \textit{name} $push\_call\_button$ and the \textit{predicate} that is determined by a closure that passes a reference to the element (\textit{origin}).
In particular, we use the value of the attribute \textit{data-floor} of the corresponding floor label (\textit{origin}) to define the event parameter (line 6).

Apply these changes by clicking on the \texttt{Reload} button and try to click on a floor label.
This should call the Event-B event \textit{push\_call\_button} with the corresponding predicate/parameter.

Let's add another interactive feature, where the user can click on the graphical element that represents the door to open or close the door respectively.
%The first step consists of registering a new Groovy function that executes the corresponding event \textit{open\_door} or \textit{close\_door}.
%Add the code snippet to your \texttt{template.groovy} file:
%\begin{lstlisting}[language=Groovy, caption={Registering a Groovy Function (Groovy)}]
%bms.registerMethod("openCloseDoor", {
%    def Trace t = bms.getTool().getTrace()
%    def Trace newTrace = executeEvent(t, "open_door", []) ?: executeEvent(t, "close_door", [])
%    if (newTrace != null) {
%        animations.traceChange(newTrace)
%        return [newState: newTrace.getCurrentState().id]
%    }
%})
%
%def Trace executeEvent(t,name,pred) {
%    try {
%        t.execute(name, pred)
%    } catch(IllegalArgumentException e) {
%        null
%    }
%}
%\end{lstlisting}
%In Line 1 we register a new Groovy function called \textit{openCloseDoor}.
%The next lines show an example how we can use the ProB Java API.
%In Line 2 we get the current trace of the animation.
%In Line 3, we first try to execute the \textit{open\_door} event by means of a helper method called \textit{executeEvent} (Line 10 to 16).
%If the return value of the helper method is null (the event could not be executed), we try to execute the event \textit{close\_door}.
%If we success (we executed one of the both events), we trigger a trace change, causing to refresh the current animation (Line 5).
%This in turn changes the state and triggers our registered observers.
%Line 6 returns a Json object that contains the state id of the new state.
%This information can used later at the JavaScript side after executing the registered \textit{openCloseDoor} method.
%Let's switch to the JavaScript side.
Add the following code snippet to your \texttt{script.js} file:
%\begin{lstlisting}[language=JavaScript, caption={Call openCloseDoor Groovy Method (JavaScript)}]
%$("#door").click(function () {
%	bms.callMethod("openCloseDoor", {
%		callback: function (data) {
%			console.log("Callback: " + data)
%		}
%	})
%}).css("cursor", "pointer")
%\end{lstlisting}
\begin{lstlisting}[language=JavaScript, caption={Interaction with the Lift Door (JavaScript)}]
bms.executeEvent({
  selector: "#door",
  events: [
    { name: "close_door" }, 
    { name: "open_door" }
  ]
});
\end{lstlisting}

This execute event handler will bind to events to the graphical element that matches the selector ``\#door''.

%In line 1 we register a new execute event handler on the graphical element with the id ``door'' (line 2).
%In line 3 to 6 we define two events.
%The execute event handler first tries to execute the event \textit{close\_door}.
%In case the event is disabled, it tries to execute the next event \textit{open\_door}.
%Line 2 calls our registered Groovy method \textit{openCloseDoor}.
%In Line 3 to 5 we pass a callback function that is called after an event (\textit{open\_door} or \textit{close\_door}) was executed.
%In this case we print the returned new state id on the console.

%These are only small examples for adding interactivity to your visualization.
%You are not limited to these examples.

%\subsection{Final visualization}

%You can download the final visualization \file{LiftVisualisation.zip}{here}.

\section{Reference}
\label{reference_b}

%\subsection{Observers}
%\label{sec:observers}
%
%Observers are used to link visual elements with the model. 
%An observer is notified whenever a model has changed its state, i.e. whenever an event has been executed. 
%In response, the observer will query the model's state and triggers actions on the linked graphical elements in respect to the new state. 
%
%\subsection{Observers in Groovy}
%\label{sec:groovy_observers}

%In general, observers are defined in the Groovy script file.
%\bms~comes with some predefined observers that are described in the following sections.

%\subsubsection{Transformer Observer}
%\label{sec:transformer_observer}

%The transformer observer supports the modification of attributes of graphical elements based on the %current state of the animation.
%The following code snippet demonstrates the basic use of the transformer observer:

%\begin{lstlisting}[float=ht,language=Groovy]
%transform("#myvisualelement") {
%    set "fill", "green"
%    set "stroke", "red"
%    register(bms)
%}
%\end{lstlisting}

%In line 1 we define a jQuery selector to select the graphical elements which should be transformed.
%In this case we select a visual element with the id \textit{myvisualelement} (in jQuery the prefix %``\#'' denotes that we want to select and element by its id).
%jQuery provides several possibilities to select graphical elements.

%\info{The jQuery selector API documentation\footnote{\url{http://api.jquery.com/category/selectors/}} provides an overview and a detailed documentation about selectors.}

%Line 2 to 3 are actions that are made on the graphical elements which are matched by the defined selector.
%In this case the \textit{fill} attribute is set to the value \textit{green} and the \textit{stroke} attribute to \textit{red}.
%In line 4, we register the observer to the current visualisation.
%A registered observer is triggered after every state change, e.g. after executing an event.
%
%\paragraph{Groovy Closures.}
%%As we use the Groovy Scripting language for defining the observers, we can make use of the entire function and feature range of it.
%Transformer observers support the use of Groovy closures for defining the selector or the value of an action.
%The following code snippet demonstrates the use of closures for transformer observers:
%
%\begin{lstlisting}[float=ht,language=Groovy]
%transform("#myvisualelement") {
%  set "fill", { (bms.eval("1 < x").value == "TRUE") ? "white" : "lightgray" }
%  set "y", {
%    switch (bms.eval("x").value) {
%      case "0": "15"
%        break
%      case "1": "20"
%        break
%      case "2": "25"
%        break
%      default: "0"
%    }
%  }
%  register(bms)
%}
%\end{lstlisting}
%
%In Groovy a closure is encapsulated in curly brackets.
%Line 2 and 3 show two examples for using closures for defining the value of the attribute \textit{fill} and \textit{y} respectively.
%Closures are evaluated after every state change in consequence of triggering the transformer observer.
%As an example, in line 2 the closure evaluates the expression $1 < x$.
%The value of the \textit{fill} attribute is set to \textit{white} whenever the expression evaluates to \textit{TRUE} and otherwise to \textit{lightgray}.
%
%
%\subsubsection{Method Observer}
%\label{sec:method_observer}
%
%The following code snippet gives an example of the method observer:
%
%\begin{lstlisting}[float=ht,language=Groovy]
%callMethod("mymethod") {
%  data([foo: "bar"])
%  register(bms)
%}
%\end{lstlisting}
%
%\subsubsection{Custom Observer}
%\label{sec:custom_observers}
%
%The following code snippet gives an example of a custom observer:
%\begin{lstlisting}[float=ht,language=Groovy]
%def customObserver = [ 
%  apply: { bms ->  println "Triggering custom observer." } 
%] as BMotionObserver
%bms.registerObserver(customObserver)
%\end{lstlisting}
%
%You can also create a custom transformer observer:
%\begin{lstlisting}[float=ht,language=Groovy]
%def transformerObserver = [
%  update: { bms ->
%    def transformers = []
%    def result = bms.translate("relation")					
%    result.value.each { obj ->
%      transformers += transform("#" + obj.first) {
%        set "transform", "translate(" + obj.second + ")"
%        update(bms)
%      }
%    }
%    return transformers
%  }
%] as BMotionTransformer
%bms.registerObserver(transformerObserver)
%\end{lstlisting}

\subsection{Observers}
\label{b_observers}

Observers are used to link graphical elements with the model. 
An observer is notified whenever a model has changed its state, i.e. whenever an event has been executed. 
In response, the observer will query the model's state and triggers actions on the linked graphical elements in respect to the new state. 

In general, observers are defined in the JavaScript file. 
\bms~comes with some predefined observers that are described in the following sections.

\subsubsection{Formula Observer}

The formula observer \textit{observes} a list of formulas (e.g.  expressions, predicates or single variables) and triggers a function whenever a state change occurred.
The values of the formulas and the origin (the reference to the graphical element that the observer is attached to) are passed to the trigger function.
Within the trigger function you can change the attributes of the origin (e.g. its colour or position) according to the values of the formulas in the respective state.
The following options can be passed to the formula observer:

\begin{itemize}

	\item[] \textbf{selector [Type: string, \textit{Required}]:}\\ A \textit{selector} that matches a set of graphical elements in the visualization (e.g. images or shapes). 
A selector follows the syntax provided by jQuery.
Fore more information about jQuery and selectors we refer the reader to the jQuery API documentation\footnote{\url{http://api.jquery.com/category/selectors/}.}. 
For instance, to match the graphical element with the ID ``elem1'' (each element should have a unique ID in the visualization) the user can define the selector ``\#elem1''.
The prefix ``\#'' is used for matching a graphical element by its ID in jQuery.

\item[] \textbf{formulas [Type: list, \textit{Required}]:}\\
Define a list of formulas (e.g. expressions, predicates or single variables) which should be evaluated in each state.
The results of the formulas are passed to the trigger function.
Example: $['x < 4', 'card(x)']$

\item[] \textbf{translate [Type: boolean, Default: false]:}\\
In general the result of the formulas are passed as strings to the trigger function.
Set this option to \textit{true} to translate B-structures to JavaScript structures.

\item[] \textbf{trigger [Type: function(origin, values), \textit{Required}]:}\\
The trigger function will be called after every state change with its \textit{origin} reference set to the graphical element that the observer is attached to and the \textit{values} of the formulas. 
The \textit{origin} is a jQuery selector element.
Consult the jQuery API\footnote{\url{http://api.jquery.com/}.} for more information regarding accessing or manipulating the \textit{origin} (e.g. set and get attributes).
The \textit{values} parameter contains the values of the defined formulas in an array, e.g. use \textit{values[0]} to obtain the result of the first formula.

The following parameters are available:

\begin{itemize}
	\item[\textbf{origin:}] The reference set to the graphical element that the observer is attached to.
	\item[\textbf{values:}] Contains the values of the defined formulas in an array, e.g. use \textit{values[0]} to obtain the result of the first formula.
\end{itemize}
 
\end{itemize}

Example formula observer:

\begin{lstlisting}[language=JavaScript]
bms.observe("formula", {
  selector: "#door",
  formulas: ["cur_floor", "door_open"],
  translate: true,
  trigger: function (origin, values) {
    switch (values[0]) {
      case -1: origin.attr("y", "275"); break
      case 0: origin.attr("y", "175"); break
      case 1: origin.attr("y", "60"); break
    }
    if(values[1]) {
      origin.attr("fill", "white");
    } else {
      origin.attr("fill", "lightgray");
    }
  }
})
\end{lstlisting}

\subsubsection{Predicate Observer}

The predicate observer accepts a predicate and applies a list of transformers depending on the result of the predicate in the current state (true or false).

\begin{itemize}
	\item[] \textbf{selector [Type: string, \textit{Required}]:}\\ A \textit{selector} that matches a set of graphical elements in the visualization (e.g. images or shapes). 
A selector follows the syntax provided by jQuery.
Fore more information about jQuery and selectors we refer the reader to the jQuery API documentation\footnote{\url{http://api.jquery.com/category/selectors/}.}. 
For instance, to match the graphical element with the ID ``elem1'' (each element should have a unique ID in the visualization) the user can define the selector ``\#elem1''.
The prefix ``\#'' is used for matching a graphical element by its ID in jQuery.
	\item[] \textbf{predicate [Type: string, \textit{Required}]:}\\ The actual predicate that should be evaluated in each state.
	\item[] \textbf{true [Type: list]:}\\ A list of transformers (\textit{attr} and \textit{value} pairs) that should be applied on the matched element whenever the defined predicate evaluates to true in the respective state. Example: 
  [\{attr: "fill", value: "white"\}, \{attr: "stroke", value: "green"\}].
	\item[] \textbf{false [Type: list]:}\\ Similar to the true option.
	However the defined list of transformers is applied whenever the predicate evaluates to false.
\end{itemize}

%*This attribute also accepts a function that should return its value.
%The reference to the graphical element that the observer is attached to (origin) is passed to the function.

Example predicate observer:

\begin{lstlisting}[language=JavaScript]
$("#door").observe("predicate", {
  predicate: "door_open",
  true: [
    {attr: "fill", value: "white"}
  ],
  false: [
    {attr: "fill", value: "lightgray"}
  ]
});
\end{lstlisting}

%\subsubsection{Refinement Observer}
%
%The refinements observer observes a list of refinements (model names) and triggers an \textit{enable} function whenever the current model is in the list of refinements or a \textit{disable} function whenever the current model is not in the list of refinements.
%The following options can be passed:
%
%\begin{tabular}{ l l l p{7cm} }
%  \textbf{Name} & \textbf{Type} & \textbf{Default} & \textbf{Description} \\
%  \hline\noalign{\medskip}
%  refinements & list / function* & empty list & Define a list of refinements (machine names) that should be observed.\\
%  \hline\noalign{\medskip}
%  enable & function &  & This function will be called after initialising or changing the model and whenever the current model is included in the list of refinements with its \textit{origin} reference set to the element that the observer is attached to.\\
%  \hline\noalign{\medskip}
%  disable & function &  & This function will be called after initialising or changing the model and whenever the current model is \textbf{NOT} included in the list of refinements with its \textit{origin} reference set to the element that the observer is attached to.\\
%\end{tabular}
%
%*This attribute also accepts a function that should return its value.
%
%\begin{lstlisting}[language=JavaScript]
%$("myvisualelement").observe("refinement", {
%  refinements: ["Machine01", "Refinement02"],
%  enable: function (origin, data) {
%    origin.attr("opacity", "1")
%  },
%  disable: function (origin) {
%    origin.attr("opacity", "0.1")
%  }
%})
%\end{lstlisting}

%\subsubsection{Remark}
%
%All JavaScript observers can also be created by means of the \textit{prob} API variable. 
%This is in particular useful, whenever the user needs to define the selector based on the result of an expression:
%
%\begin{lstlisting}[language=JavaScript]
%prob.observe("formula", {
%  formulas: ["myvar", "floor", "x>4"],
%  trigger: function (res) {
%    var el = $("#" + res[0])
%    el.html(res[1])
%    if(res[2] === "TRUE") {
%      el.attr("fill","green")
%    } else {
%      el.attr("fill","red")
%    }
%  }
%});
%\end{lstlisting}

\pagebreak

\subsection{Event Handler}
\label{b_event_handler}

\subsubsection{Execute Events}

The execute event handler binds a click handler that executed an event (Event-B) or an operation (Classical-B) on the element(s) that matches the selector.
The user can also define a list of events (or operations).
In this case, a tooltip that lists the available events (disabled and enabled) will be shown when hovering the matched element.
The following options can be passed:

\begin{itemize}

\item[] \textbf{selector [Type: string, \textit{Required}]:}\\ A \textit{selector} that matches a set of graphical elements in the visualization (e.g. images or shapes). 
A selector follows the syntax provided by jQuery.
Fore more information about jQuery and selectors we refer the reader to the jQuery API documentation\footnote{\url{http://api.jquery.com/category/selectors/}.}. 
For instance, to match the graphical element with the ID ``elem1'' (each element should have a unique ID in the visualization) the user can define the selector ``\#elem1''.
The prefix ``\#'' is used for matching a graphical element by its ID in jQuery.

\item[] \textbf{events [Type: list, \textit{Required}]:}\\
Define a list of events with \textit{name} and \textit{predicate} that should be bind with the matched graphical elements.

\begin{itemize}

\item[] \textbf{name [Type: string, \textit{Required}]:}\\
The name of the event. 
If the value is a function it takes the return value of the function.

\item[] \textbf{predicate [Type: string*]:}\\
The predicate that defines the parameters of the event to be executed.
If the value is a function it takes the return value of the function.

\end{itemize}

\item[] \textbf{label [Type: function(event, origin)]:}\\
A function that returns a custom label (string) to be shown in the tooltip.
You can also return an HTML element, e.g. \textit{span} or \textit{strong}.
 The function provides two arguments: \textit{event} a json object containing some data of the respective event and \textit{origin} as the reference to the graphical element where the execute event handler is attached to.
	
\end{itemize}

*This attribute also accepts a function that should return its value.
The reference to the graphical element that the observer is attached to (origin) is passed to the function.

%\vspace{0.5cm}
%\begin{tabular}{ l l l p{7cm} }
%  \textbf{Name} & \textbf{Type} & \textbf{Default} & \textbf{Description} \\
%  \hline\noalign{\medskip}
%  selector & string / function* & & jQuery selector \\
%  \hline\noalign{\medskip}  
%  events & list & empty list & Define a list of events with \textit{name} and \textit{predicate} that should be bind with the matched graphical elements. \\
%  \hline\noalign{\medskip}
%  : name & string / function* & & The name of the event. If the value is a function it takes the return value of the function.\\
%  \hline\noalign{\medskip}
%  : predicate & string / function* & & The predicate that defines the parameters of the event to be executed. If the value is a function it takes the return value of the function.\\
%  \hline\noalign{\medskip}
%  tooltip & boolean & true & Enable (\textit{true}) or disable (\textit{false}) tooltip when hovering the matched element.\\
%  \hline\noalign{\medskip}
%  callback & function &  & The callback function is called whenever a bind event was executed.
%\end{tabular}
%
%*This attribute also accepts a function that should return its value.

Example execute event handler:

\begin{lstlisting}[language=JavaScript]
bms.executeEvent({
  selector: "text[data-some]",
  events: [
    { 
      name: "event1", 
      predicate: function (origin) {
        return "x=" + origin.attr("data-some") 
      }
    },
    { name: "event2", predicate: "y=3" },
    { name: "event3" } 
  ],
  label: function(event, origin) {
     return "<strong>" + event.name + "." + event.predicate + "</strong>";
  }
});
\end{lstlisting}


%\chapter{BMotionWeb for CSP}
%\label{bms4csp}
%\section{Tutorial}
\label{tutorial_csp}

The objective of this chapter is to get you to a stage where you can use BMotion Studio to visualize CSP-M models. 
We expect that you have already downloaded the BMotion Studio tool (see Section~\ref{installation}) and created a new visualisation template for Event-B visualisations (see Section~\ref{vis_template}).
 
%We expect you to have a basic understanding of logic and an idea why doing formal modelling is a good idea.  
You should be able to work through the tutorial with no or little outside help.

We encourage you not to download solutions to the examples but instead to actively build them up yourself as the tutorial progresses.

If something is unclear, remember to check the Reference (\ref{reference_csp}) for more information.

\subsection{Creating a new Visualisation Template}

Let's start by creating a new visualisation template.
The easiest way to create a new visualisation template is to duplicate the default template \texttt{csp\_template} that is included in the \texttt{workspace} folder of your \bms~installation.
Just duplicate the \texttt{csp\_template} folder and rename it to \texttt{bully}.
After refreshing your browser, a new folder called \texttt{bully} should appear in your workspace.

\subsection{The Formal Model}

We are going to create a visualisation of the bully algorithm specification found in the book ``Understanding Concurrent Systems'' (ISBN 978-1-84882-258-0).
The algorithm represents a method of distributed computing for electing a node to be the coordinator amongst a group of nodes.
Each node has a unique ID and the algorithm intends to select the node with the highest ID to be the coordinator.
It is assumed that the nodes may fail and revive from time to time and the communication between the nodes is reliable.
Three types of messages are defined within the design of the algorithm: \textit{election} (announcing an election), \textit{answer} (responding to an election message), and \textit{coordinator} (announcing the identity of the coordinator).

You can download the CSP model \file{BullyAlgorithm.zip}{here}.
Decompress it and put the files into a new folder called \texttt{model} relative to your \texttt{template.html} file in your workspace.

\subsection{Linking a Model with the Visualisation}

The next step consists of linking the model with the visualisation.
For this, open the \texttt{template.html} file with an HTML/text editor of your choice and add the following line within the head tag:

\begin{lstlisting}[language=html]
<meta name="bms.model" content="model/bully.csp" />
\end{lstlisting}

We link the visualisation with the CSP-M model called ``bully.csp''.
Linking a model within the \texttt{template.html} file automatically loads the model, when starting the visualisation.
Your \texttt{template.html} file should look like:

\begin{lstlisting}[language=html]
<html bms-app>
  <head>
      <title>BMotion Studio for ProB</title>
      <meta name="bms.model" content="model/bully.csp" />
      <meta name="bms.tool" content="CSPAnimation" />
      <meta name="bms.script" content="template.groovy" />
      <script src="/bms/libs/requirejs/require.js"></script>
      <script>
        require(['/bms/libs/prob/config.js'], function () {
          require(['template']);
        });
      </script>
  </head>
  <body>
  </body>
</html>
\end{lstlisting}

\info{The meta tag \textit{bms.script} (line 6) contains the link to the Groovy script file and the meta tag \textit{bms.tool} (line 5) defines the formalism or the simulator respectively that should be used. In this case we are creating a visualisation for a ``CSPAnimation'' (CSP models).}

\subsection{Creating the Actual Visualisation}

The next step consists of creating the actual visualisation.
The user is not restricted to an editor in order to create a visualisation.
The user can make use of any tool that support the creation of SVG graphics or HTML documents.
For this tutorial we are going to use the Inkspace\footnote{\url{https://inkscape.org}} editor. Inkscape is an editor for creating vector graphics that is available for Windows, Mac OS X and Linux.
It's free and open source.
With Inkscape the user can export the vector graphic into the SVG format.

%\info{We are currently working on a build-in graphical editor for creating SVG graphics and for managing observers.}

\begin{figure}[!ht]
\begin{center}
	\includegraphics[width=12cm]{img/tutorial/tut_02.png}
	\caption{Creating an SVG graphic with Inkscape}
	\label{fig_tut_02_inkscape}
\end{center}
\end{figure}

%Please download the prepared \file{lift.svg}{lift.svg} file and put it relative to your \texttt{template.html} file in your workspace.
%Add the following tag within the body tag in your \texttt{template.html} file:
%\begin{lstlisting}[language=html]
%<object data="lift.svg" type="image/svg+xml" data-bms="svg">
%</object>
%\end{lstlisting}

Please download the prepared \file{lift.svg}{lift.svg} file and open it with Inkscape as demonstrated in Figure~\ref{fig_tut_02_inkscape}.
Feel free to modify and explore the SVG graphic.
In order to link visual elements of the SVG graphic with the formal model, we have to give them identifiers. 
For this, select an element, open the context menu and select \textsf{Object Properties}.
A popup window should be opened as demonstrated in Figure~\ref{fig_tut_02_inkscape}.
As an example, we give the visual element that represents the door, the id ``door''.
In Section~\ref{sec_creation_observers} we explain how we can use this information in order to create the link between the formal model and the visualisation by means of observers.
If you are satisfied with your SVG graphic, save it as a plain SVG graphic with \textsf{File $\rangle$ Save As}.
Select \textsf{Plan SVG (*.svg)} as an output format and click on the \textsf{Save} button.
You can save the SVG file anywhere on your local system. 
Open the SVG file with a text editor of your choice and put the SVG code within the body tag in the \texttt{template.html} file located in your workspace.

\section{Reference}
\label{reference_csp}


\chapter{Frequently Asked Questions}
\label{faq}

\section{How to Build a Standalone Visualisation}

Clone the BMotion Studio for ProB standalone Github repository\footnote{\url{https://github.com/ladenberger/bmotion-prob-standalone}} and put your visualisation into the folder ``resources/workspace''.
In the root folder run the following command, where XXX is the path to the html template file in the ``resources/workspace'' folder (e.g. ``myvis/vis.html''):
\begin{lstlisting}[language=bash]
gradle -Pvisualisation="XXX" buildAll
\end{lstlisting}

If you don't have gradle installed, you can use the gradlew script provided:
\begin{lstlisting}[language=bash]
./gradlew -Pvisualisation="XXX" buildAll
\end{lstlisting}

This should build the binaries without a gradle installation on your computer.
The gradle script will produce a zipped standalone version for all platforms. The zip files are located in the build/distributions folder.





\clearpage
\phantomsection
\addcontentsline{toc}{chapter}{Index} 
\printindex

\end{document}

