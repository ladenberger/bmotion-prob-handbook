
The objective is to get you to a stage where you can use BMotion Studio to visualize Event-B models.  We expect you to have a basic understanding of logic and an idea why doing formal modeling is a good idea.  You should be able to work through the tutorial with no or little outside help.

This tutorial covers installation and configuration of BMotion Studio; it brings you through step by step through building visualizations for formal models and it provides the essential theory and provides pointers to more information.

We attempt to alternate between theory and practical application thereby keeping you motivated.  We encourage you not to download solutions to the examples but instead to actively build them up yourself as the tutorial progresses.

If something is unclear, remember to check the Reference (\ref{reference}) for more information.

\section{Outline}

\begin{description}
	\item[Background before getting started (\ref{tutorial_01})] We give a brief description of what BMotion Studio is and what it is being used for and what kind of background knowledge we expect.
	\item[Installation and first steps (\ref{tutorial_02})] We guide you through downloading, installing and starting BMotion Studio. We name the visible views and describe what they are doing. We introduce a first visualization for a waterboiler model. We explain how to create a new visualization and describe briefly the edit perspective.
	\item[Working with Controls (\ref{tutorial_03})] We explain the concept of a control.
	\item[Working with Observer (\ref{tutorial_04})] The behavior of controls is determined by observers. We explain you the concept of an observer.
	\item[Working with Events (\ref{tutorial_05})] Controls can execute events. We describe how to assign and configure events.
	\item[Starting a Visualization (\ref{tutorial_06})] We show you how to start and to work with a visualization.
\end{description}

