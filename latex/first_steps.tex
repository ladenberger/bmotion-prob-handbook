\section{Installation and Start}
\label{installation}

Start off by downloading \bms~for your operating system.
You can find the latest version of the tool at \url{http://www.stups.hhu.de/ProB/index.php5/BMotionWeb}.
Decompress the archive and expand the directory if necessary.
%\warning{Do not change the location and structure of the files and directories within the folder!}
%Navigate to the \texttt{server/bin} folder and start the server by entering the bash command:
%
%\begin{lstlisting}[language=bash]
%.\standalone
%\end{lstlisting}
%
%\info{Windows users should execute the \texttt{standalone.bat} file.}
Navigate to the application folder and start \bms~by executing the \texttt{bmotion-prob} binary.
%Now, navigate to the \texttt{client} folder and start the client by executing the \texttt{bmotion-prob} program.
After a short loading time you should see the window shown in Figure~\ref{fig_bms_client}.

\begin{figure}[!ht]
\begin{center}
	\includegraphics[width=.8\textwidth]{img/tutorial/clientstartscreen.png}
	\caption{\bms\ Client}
	\label{fig_bms_client}
\end{center}
\end{figure}

%Your default browser should open and show the default workspace.
%The workspace contains the following predefined folders:
%\begin{itemize}
%\item \texttt{libs}: This folder contains JavaScript libraries that are needed for BMotionWeb.
%\item \texttt{b\_template}: A visualization template for creating visualizations for Classical-B and Event-B models.
%\item \texttt{csp\_template}: A visualization template for creating visualizations for CSP models.
%\end{itemize}

\section{Open a Visualization}
\label{open_vis_template}

To open a visualization, click on the box in the middle of the window and select the \bms\ manifest file (see \ref{sec:manifest}) of the visualization or just drag and drop the \bms\ manifest file into the box.
You can also open a visualization via the top menu: \textsf{File $\rangle$ Open Visualization}.

\section{Create a new Visualization}
\label{vis_template}

To create a new visualization choose \textsf{File $\rangle$ New Visualization}.
A window will be opened asking you for some additional information (e.g. the id and name of your visualization).
Enter your data and press on OK.
Now, the tool asks you for a location (a folder) where you want to save your visualization.
In the next step, the tool asks you for a model to be visualized.
Select a model and click on OK.
This will start the fresh visualization.
The tool will create a bunch of files into the selected folder:

%Let's start by creating a new visualization template.
%You can download the \file{bms-b-template.zip}{predefined template} as a starting point to create a new visualization template.
%The easiest way to create a new visualization template is to download the \file{bms-b-template.zip}{predefined template}.
%Decompress the archive, expand the directory if necessary and navigate to the decompressed folder.

%	duplicate one of the default templates  \texttt{b\_template} (for Event-B or Classical-B visualizations) or \texttt{csp\_template} (for CSP-M visualizations) that are included in the \texttt{workspace} folder of your \bms~installation.
%Just duplicate the folder.
%After refreshing your browser, the newly created folder should appear in your workspace.
%Navigate to the folder.
%The folder contains three files.

%\paragraph{\texttt{template.groovy:}}
%The Groovy script file is the place where the user can communicate with the formal model by means of the ProB Java API\footnote{\url{http://www.stups.hhu.de/ProB/index.php5/ProB_Java_API}}.
%For instance, the user may register methods that can be called in the JavaScript w.
%The Groovy script file is the place where you can setup the communication between your visualization and the ProB animator.
%In particular, the Groovy script file is the link between the formal model and the visualization.
%It allows you to programmatically control the ProB animator and to access the actual formal model being visualised.

\paragraph{\texttt{bmotion.json:}}
The \texttt{bmotion.json} file is the root file of your \bms\ visualization (also called \bms\ manifest file).
It contains the configuration formatted using JSON (JavaScript Object Notation)\footnote{\url{http://www.json.org}.}.

\info{Section~\ref{sec:manifest} contains a full list of available options.}

%A minimal configuration of the manifest file should contain the path to the formal model you want to visualize and at least one view configuration.
%A minimal configuration of the manifest file should contain at least one view configuration.

Example manifest file with a minimal configuration:

\begin{lstlisting}[language=JavaScript]
{
    "id": "lift",
    "name": "Lift visualization",
    "template": "index.html",
    "model": "model/lift.mch"
}
\end{lstlisting}


\paragraph{\texttt{script.js:}}
In the JavaScript file you can setup observers and actions (see Section~\ref{reference_b}).
Moreover, the user can take advantage of the entire JavaScript language.
There exist are a lot of libraries for JavaScript that you can apply to create custom visualizations.
For instance, it exists libraries for generating chart and plot diagrams.
%In addition, you can call functions that are registered in the Groovy script file.
%This enables you to add some interactivity to your visualization.
%For instance, pressing a button in your visualization could cause the execution of an Event-B event.

%The \textit{prob} parameter is the access point to the BMotionWeb for ProB API.

\paragraph{\texttt{index.html:}}
%The HTML file is the root file of your visualization. It contains the actual visualization.
The HTML file contains the reference to the \texttt{scripts.js} file and to the \texttt{visualization.svg} file.

Example \texttt{index.html} file with a minimal configuration:

\begin{lstlisting}[language=html]
<html>
  <head>
    <title>BMotionWeb Visualization</title>
  </head>
  <body>
    <script src="bms.api.js"></script>
    <script src="script.js"></script>
    <div data-bms-svg="visualization.svg"></div>
  </body>
</html>
\end{lstlisting}

%Please note the empty attribute \textit{data-bms-visualisation} in line 1 and the \textit{script} tag in line 6.
%Every \bms\ visualization template file should contain the empty attribute \textit{data-bms-visualisation} (line  1) and a reference to your \texttt{scripts.js} (line 6).

\paragraph{\texttt{visualization.svg:}}

%The meta tag \textit{bms.tool} (line 4) defines the formalism or the simulator respectively that should be used.
%Two values are allowed: ``BAnimation'' for creating visualizations of Event-B or Classical-B models and ``CSPAnimation'' for creating visualizations of CSP-M models.
%The meta tag \textit{bms.script} (line 5) contains the link to the Groovy script file.
%Finally, in line 9 we define the path to the JavaScript file.

The actual SVG visualization.
The user is not restricted to an editor in order to create a visualization.
The user can make use of the integrated visual editor or any other tool that supports the creation of SVG graphics.

\paragraph{\texttt{bms.api.js:}}
JavaScript library that is needed for running the visualization.
Please do not edit this file!

%\section{Link a Model with a Visualization}
%
%In order to link a model with the visualization, open the \texttt{bmotion.json} file with an editor of your choice and adapt the \textit{model} property.
%The model \textit{model} property should contain the path to your model file (e.g. \texttt{mymodel/model.mch}).
%The model should be places relative to your bmotion.json file.
%Linking a model within the \texttt{bmotion.json} file will automatically load the model, when starting the visualization.

%To create a link between graphical elements and the model, please checkout the Section \ref{tutorial_b} for Event-B and Classical-B or \ref{tutorial_csp} for CSP.
