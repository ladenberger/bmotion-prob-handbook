\section{Before Getting Started}
\label{tutorial_01}

Before we get started with the actual tutorial, we are going to go over the required background to make sure that you have a rudimentary understanding of the necessary concepts.

\tick{\textbf{You can skip this section, if...}
\begin{itemize}
	\item ... you know what formal modelling is
	\item ... you know what predicate logic is
	\item ... you know what ProB refer to
	\item ... you know what Eclipse is
\end{itemize}
}

\subsection{Formal Modelling}

We are concerned with \textit{formalizing specifications}.  This allows us a more rigorous analysis (thereby improving the quality) and allows the reuse of the specification to develop an implementation.  This comes at the cost of higher up-front investments.

This differs from the traditional development process. In a formal development, we transfer some effort from the test phase (where the implementation is verified) to the specification phase (where the specification in relation to the requirements is verified).

\subsection{Predicate Logic}
\label{predicate_logic}
\index{predicate logic}

Predicate logic is a mathematical logic containing variables that can be quantified.

Event-B supports first-order logic which is, technically speaking, just one type of predicate logic.  

\subsection{Event-B}
\label{eventb}
\index{Event-B}

Event-B is a notation for formal modelling based around an abstract machine notation (\index{abstract machine notation}).

Event-B is considered an evolution of B (also known as classical B). It is a simpler notation which is easier to learn and use. It comes with tool support in the form of the Rodin Platform.

\subsection{ProB Animator}
\label{prob_animator}

ProB is a validation toolset for the B method including an animator, a modelchecker and other useful tools to allow users to gain confidence in their specifications. One of the components of ProB is animation. The animation component allows the user to check the presence of desired functionality and to inspect the behaviour of a specification by "clicking through" the states of the specification. ProB also provides other useful tools such as a tool to visualize graphically any predicate as a tree or a tool for graphical state representation. Such tools, especially the tool for the graphical state representation can give a better understanding of the model.

\subsection{Eclipse}

BMotion Studio is based on the Eclipse Platform (\ref{eclipse}), a Java-based platform for building software tools.  This matters for two reasons:
\begin{itemize}
	\item If you have already used Eclipse-based software, then you will feel immediately comfortable with the handling of the BMotion Studio application.
	\item Many extensions, or plugins, are available for Eclipse-based software. So it is possible to create plug-ins for BMotion Studio using the extension mechanismus of Eclipse.
\end{itemize}

The GUI of an Eclipse application consists of Views, Editors, Toolbars, Quickview, Perspectives and many more elements.  If these terms are unfamiliar to you, please consult Section~\ref{eclipse} which contains references to Eclipse tutorials.
